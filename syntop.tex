\documentclass{article}
\input{decls}
\usepackage[status=draft]{fixme}
\title{Unifying constructive general topology}
\author{Toby Bartels and Michael Shulman}
\def\Q{\mathbb{Q}}
\def\Qp{\mathbb{Q}^+}
\def\R{\mathbb{R}}
\def\Re{\overline{\R}}
\def\Rpd{[0,\infty]_d}
\def\Rpm{[0,\infty]_m}
\def\Rpu{[0,\infty]_u}
\def\Rpl{[0,\infty]_{\ell}}
\def\Rp{[0,\infty]}
\def\Red{\Re_d}
\def\Rem{\Re_m}
\def\Reu{\Re_u}
\def\Rel{\Re_{\ell}}
\def\upp#1{{#1}^{\sharp}}
\def\low#1{{#1}^{\flat}}
\def\U[#1]{\mathsf{U}[#1]}
\def\L[#1]{\mathsf{L}[#1]}
\def\F{\mathcal{F}}
\def\apart{\mathrel{\#}}
\def\napprox{\not\approx}
\def\nle{\not\le}
\def\oapt{\mathrel{\!\not\,\not\lesssim}}
\def\leapx{\lesssim}
\def\ent#1{\leapx_{#1}}
\def\aent#1{\oapt_{#1}}
\def\int{\mathrm{int}}
\def\ext{\mathrm{ext}}
\def\cl{\mathrm{cl}}
\def\cpl#1{\neg #1}
\let\implies\Rightarrow
\def\inv{^{-1}}
\def\hfep{\frac{\ep}{2}}
\def\fep#1#2{\frac{\if1#1\else#1\fi\ep}{#2}}
\def\singleton#1{\{#1\}}
\def\anti{\mathfrak{A}}
\def\neigh{\mathfrak{Z}}
\def\nn{\ensuremath{\neg\neg}}
\def\sat#1{\widecheck{#1}}
% Categories
\def\Set{\mathbf{Set}}
\def\PMet{\mathbf{PMet}}
\def\QGau{\mathbf{QGau}}
\def\Met{\mathbf{Met}}
\def\MetTop{\mathbf{Met}_{\mathbf{Top}}}
\def\PMetTop{\mathbf{PMet}_{\mathbf{Top}}}
\def\PMetUnif{\mathbf{PMet}_{\mathbf{Unif}}}
\def\PMetProx{\mathbf{PMet}_{\mathbf{Prox}}}
\def\PMetApp{\mathbf{PMet}_{\mathbf{App}}}
\def\PMetOrd{\mathbf{PMet}_{\mathbf{Ord}}}
\def\PMetu{\mathbf{PMet}_u}
\def\PMetTopu{\mathbf{PMet}_{\mathbf{Top},u}}
\def\PMetUnifu{\mathbf{PMet}_{\mathbf{Unif},u}}
\def\PMetProxu{\mathbf{PMet}_{\mathbf{Prox},u}}
\def\PMetAppu{\mathbf{PMet}_{\mathbf{App},u}}
\def\PMetOrdu{\mathbf{PMet}_{\mathbf{Ord},u}}
\def\TopPMet{\mathbf{TopPMet}}
\def\UnifPMet{\mathbf{UnifPMet}}
\def\ProxPMet{\mathbf{ProxPMet}}
\def\AppPMet{\mathbf{AppPMet}}
\def\OrdPMet{\mathbf{OrdPMet}}
\def\SatPMet{\mathbf{SatPMet}}
\def\Preord{\mathbf{Preord}}
\def\APreord{\mathbf{APreord}}
\def\PTop{\mathbf{pTop}}
\def\Top{\mathbf{Top}}
\def\PTopnn{\mathbf{pTop}_{\nn}}
\def\ldTop{\mathbf{Top}_{\mathbf{ld}}}
\def\APTop{\mathbf{pATop}}
\def\ATop{\mathbf{ATop}}
\def\APTopnn{\mathbf{pATop}_{\nn}}
\def\ldATop{\mathbf{ATop}_{\mathbf{ld}}}
\def\QUnif{\mathbf{QUnif}}
\def\Unif{\mathbf{Unif}}
\def\Unifnn{\mathbf{Unif}_{\nn}}
\def\ldUnif{\mathbf{Unif}_{\mathbf{ld}}}
\def\ldQUnif{\mathbf{QUnif}_{\mathbf{ld}}}
\def\AUnif{\mathbf{AUnif}}
\def\AUnifnn{\mathbf{AUnif}_{\nn}}
\def\ldAUnif{\mathbf{AUnif}_{\mathbf{ld}}}
\def\AQUnif{\mathbf{AQUnif}}
\def\AUnifnn{\mathbf{AUnif}_{\nn}}
\def\ldAQUnif{\mathbf{AQUnif}_{\mathbf{ld}}}
\hyphenation{pro-metric}
\begin{document}
\maketitle

\section{Introduction}
\label{sec:intro}

Guiding principles:
\begin{enumerate}
\item Category theory.  Large categories should be well-behaved, constructions should be functorial and natural and related by universal properties.
\item Nice categories including not-nice objects.  In particular, only impose separation axioms when needed, and include trivial cases like the empty set.
\item There's nothing intrinsically wrong with points, although they don't tell the whole story.
\item There's nothing intrinsically wrong with negation, as long as it is used correctly.
  For instance, $x\le y$ can be defined for Dedekind real numbers as $\neg(y<x)$.
\item Topos-valid mathematics.  Equality means equality; it doesn't have to be ``proved'' that a function respects equality.  Sets don't come with a God-given apartness relation; indeed an apartness is a sort of topological structure that a set can be \emph{equipped} with.
\end{enumerate}

\section{Real numbers}
\label{sec:reals}

As is well-known, constructively there are many different kinds of ``real numbers''.
Since the properties of metric (and gauge and prometric) spaces vary depending on which kind of real numbers we use as distance, we begin by reviewing all the relevant kinds of real number.

Let $\Q$ denote the rational numbers and $\Qp = \setof{ q\in \Q | q>0}$ the set of positive rational numbers.

\begin{defn}
  An \textbf{extended upper real number} is a set $U\subseteq \Q$ such that
  \begin{enumerate}
  \item $y\in U$ if and only if there is an $x\in U$ with $x<y$ (upper-rounded).
  \end{enumerate}
  Similarly, an \textbf{extended lower real number} is a let $L\subseteq \Q$ such that
  \begin{enumerate}
  \item $x\in L$ if and only if there is a $y\in L$ with $x<y$ (lower-rounded).
  \end{enumerate}
  We write $\Reu$ and $\Rel$ for the sets of extended upper and lower real numbers, respectively.
\end{defn}

The word \emph{extended} indicates that $\pm\infty$ are included.
In $\Reu$, we have $+\infty = \emptyset$ and $-\infty = \Q$, while in $\Rel$ the meanings are reversed.

For $U_1,U_2\in \Reu$ we define
\begin{itemize}
\item $U_1 \le U_2$ if $U_2\subseteq U_1$.
\item $U_1 < U_2$ if $\exists x\in \Q, (x\in U_1 \land x\notin U_2)$.
\item $U_1 + U_2 = \setof{x+y | x\in U_1 \land y\in U_2}$.
\end{itemize}
Dually, for $L_1,L_2 \in\Rel$ we define
\begin{itemize}
\item $L_1 \le L_2$ if $L_1\subseteq L_2$.
\item $L_1 < L_2$ if $\exists x\in \Q, (x\in L_2 \land x\notin L_1)$.
\item $L_1 + L_2 = \setof{x+y | x\in L_1 \land y\in L_2}$.
\end{itemize}
In both cases, both inequalities are transitive, $\le$ is reflexive and antisymmetric, $<$ is irreflexive and a module over $\le$ on both sides.
Moreover, addition is a commutative monoid that respects both inequalities; the unit of $\Reu$ is $0_u = \setof{q\in \Q | q > 0}$ and that of $\Rel$ is $0_\ell = \setof{q\in \Q | q < 0}$.
In particular, both $\Reu$ and $\Rel$ are monoidal posets.

If $U\in\Reu$, we define its \textbf{complementary lower real number} to be
\[ \low U = \setof{x\in\Q | \exists y\in Q, (x<y \land y\notin U)}. \]
Similarly, if $L\in\Rel$ we define its \textbf{complementary upper real number} to be
\[ \upp L = \setof{y\in\Q | \exists x\in Q, (x<y \land x\notin L)}. \]

\begin{thm}
  $\low{(-)}$ and $\upp{(-)}$ both preserve $\le$ and $<$, and we have a normal monoidal adjunction of posets
  \[ \upp{(-)} : \Rel \rightleftarrows \Reu : \low{(-)}.\]
  In particular, $\upp{(-)}$ is normal colax monoidal and $\low{(-)}$ is normal lax monoidal, i.e.\ $\upp{(x+y)} \le \upp{x}+\upp{y}$ and $\low{x}+\low{y}\le \low{(x+y)}$, while both preserve $0$.
\end{thm}
\begin{proof}
  If $U_1 \le U_2$, so that $U_2\subseteq U_1$, and if $x\in\low{U_1}$, we have $y\notin U_1$ such that $x<y$.
  But then $y\notin U_2$, so $x\in \low{U_2}$ as well; thus $\low{U_1}\le \low{U_2}$.

  If $U_1 < U_2$ we have an $x$ such that $x\in U_1$ and $x\notin U_2$.
  But then since $U_1$ is rounded, we have a $y\in U_1$ with $y<x$, hence $y\notin U_2$ since $U_2$ is rounded; thus $y \in \low{U_2}$ and $y\notin \low{U_1}$, so $\low{U_1} < \low{U_2}$.

  By definition, $\low{U_1} +\low{U_2}$ is the set of sums $x_1+x_2$ such that there exist $y_1,y_2$ with $x_1<y_1$ and $x_2<y_2$ and $y_1\notin U_1$ and $y_2\notin U_2$.
  To show $\low{U_1} +\low{U_2} \subseteq \low{(U_1+U_2)}$, therefore, suppose given such $x_1,x_2,y_1,y_2$, and set $z=x_1+x_2$ and $w = y_1+y_2$.
  Certainly $z<w$, so to prove $z\in \low{(U_1 + U_2)}$ it remains to prove that $w\notin U_1+U_2$.
  Thus, suppose $w = w_1+w_2$ with $w_1\in U_1$ and $w_2\in U_2$; then by properties of addition in $\Q$, either $y_1\ge w_1$ or $y_2\ge w_2$, hence either $y_1\in U_1$ or $y_2\in U_2$, which are both contradictions.

  Finally, $\low{0_u}$ is the set of $x\in \Q$ such that there is a $y>x$ with $y\le 0$, which is to say the set of all $x\in \Q$ such that $x<0$, i.e.\ $0_\ell$.
  Thus $\low{(-)}$ is normal lax monoidal.

  The proofs of all the analogous properties for $\upp{(-)}$ are dual, so it remains to check the adjunction.
  To say $\upp L \le U$ is to say $U \subseteq \upp L$, i.e.\ for all $y\in U$ there exists an $x\in \Q$  with $x<y$ and $x\notin L$.
  This certainly implies $U \subseteq \Q\setminus L$.
  But conversely, if $U \subseteq \Q\setminus L$, then for any $y\in U$ there is an $x\in U$ with $x<y$, hence $x\notin L$.
  Thus $\upp L \le U$ is equivalent to $U \subseteq \Q\setminus L$, i.e.\ to $L\cap U = \emptyset$.
  Similarly, $L \le \low U$ is also equivalent to $L\cap U =\emptyset$. so they are equivalent to each other.
\end{proof}

Since any adjunction between posets is idempotent, this adjunction restricts to an isomorphism between the images of $\low{(-)}$ and $\upp{(-)}$.
This common fixed poset is denoted $\Rem$, and its elements are called the \textbf{extended MacNeille real numbers}.
They are equivalently definable as pairs $(L,U)$ of a lower and an upper extended real number such that
\begin{align*}
 L &= \setof{x\in\Q | \exists y\in \Q, (x<y \land y\notin U)}\\
 U &= \setof{y\in\Q | \exists x\in \Q, (x<y \land x\notin L)}.
\end{align*}
Finally, an \textbf{extended Dedekind real number} is a similar pair $(L,U)$ such that
\begin{center}
  For any $x,y\in\Q$, if $x<y$, then either $x\in L$ or $y\in U$.
\end{center}
We write $\Red$ for the set of extended Dedekind real numbers; evidently $\Red \subseteq \Rem$.

If $M_1,M_2\in\Rem$, then $M_1 \le M_2$ is equivalent to $\neg(M_2<M_1)$.
Since $M_1=M_2$ is equivalent to $(M_1 \le M_2) \land (M_2 \le M_1)$, it follows that equality in $\Rem$ (and hence also in $\Red$) is \nn-stable.
But even for Dedekind reals, $D_1<D_2$ is strictly stronger than $\neg(D_2 \le D_1)$, and $D_1\le D_2$ is strictly weaker than $(D_1=D_2) \lor (D_1 < D_2)$.

For MacNeille and Dedekind reals (unlike upper and lower reals), we can define subtraction: the additive inverse of $(L,U)$ is $( \setof{-x | x\in U}, \setof{-y | y\in L})$.
Thus $\Rem$ and $\Red$ are ordered abelian groups.
(In fact they are even rings, and ``fields'' in (different) appropriate senses, but we will not have any use for their multiplications.)

Finally, we note that the union of any number of upper real numbers (as subsets of $\Q$) is again such.
Thus, $\Reu$ is a complete lattice under $\le$, with infima constructed as unions.
Its therefore also has suprema, though these are not set-theoretic intersections but rather their interiors:
\[ \bigvee U_i = \setof{ y\in\Q | \exists x\in\Q, (x<y \land \forall i, x\in U_i}\]
Similarly, $\Rel$ has suprema constructed as unions and infima as the interiors of intersections.

Since $\Rem$ is a coreflective sub-poset of $\Reu$ and a reflective sub-poset of $\Rel$, it is closed under suprema in $\Reu$ and infima in $\Rel$.
Thus, it is also a complete lattice under $\le$, with suprema constructed as the interiors of intersections of upper sets, and infima constructed as the interiors of intersections of lower sets.
However, $\Red$ is \emph{not} in general a complete lattice.

Classically, all four kinds of real number coincide.
By contrast, the extreme constructive differences between them are illustrated by the following observations.

\begin{defn}
  Let $P$ be any truth value; we define
  \begin{align*}
    \U[P] &= \setof{x\in \Q | (x>0) \land P}\\
    \L[P] &= \setof{x\in \Q | (x<0) \lor P}
  \end{align*}
\end{defn}

Evidently $\U[P] \in \Reu$ and $\L[P]\in \Rel$, and both are $\ge 0$.
Note also that
\[\U[\top] = \L[\bot] = 0 \qquad\text{and}\qquad \U[\bot] = \L[\top] = +\infty.\]

\begin{thm}\label{thm:up}
  For any truth values $P$, $Q$, and $R$, we have
  \begin{gather}
    (\U[P] = \U[Q]) \iff (P\leftrightarrow Q)\label{eq:up1}\\
    (\L[P] = \L[Q]) \iff (P\leftrightarrow Q)\label{eq:up2}\\
    (\U[P] = 0) \iff P\label{eq:up3}\\
    (\L[P] > 0) \iff P\label{eq:up4}\\
    \upp{\L[P]} = \U[\neg P]\label{eq:up5}\\
    \low{\U[P]} = \L[\neg P]\label{eq:up6}\\
    (\L[Q] \le \L[P]) \iff (\U[P] \le \U[Q]) \iff (Q\implies P)\label{eq:up7}\\
    (\L[Q] < \L[P]) \iff (\U[P] < \U[Q]) \iff (P \land \neg Q)\label{eq:up8}\\
    (\U[P] \le \U[Q] + \U[R]) \iff (Q\land R \implies P)\label{eq:up9}\\
    (\L[Q] \le \L[P] + \L[R]) \iff (Q \implies P\lor R)\label{eq:up10}
  \end{gather}
  Moreover, $\U[P]$ and $\L[P]$ are MacNeille if and only if $\neg\neg P\implies P$, and Dedekind if and only if $P\lor\neg P$.
\end{thm}
\begin{proof}
  Statements~\eqref{eq:up1} and~\eqref{eq:up2} are clear.
  Note that~\eqref{eq:up1} implies~\eqref{eq:up3}, while~\eqref{eq:up7} also implies~\eqref{eq:up3} (take $Q=\top$).
  Similarly,~\eqref{eq:up8} implies~\eqref{eq:up4} (take $Q=\bot$).
  Moreover,~\eqref{eq:up9} and~\eqref{eq:up10} together imply~\eqref{eq:up7} (take $R=\top$ in~\eqref{eq:up9} and $R=\bot$ in~\eqref{eq:up10}).
  Thus, it remains to prove~\eqref{eq:up5},~\eqref{eq:up6},~\eqref{eq:up8},~\eqref{eq:up9}, and~\eqref{eq:up10}, plus the final statement about MacNeille and Dedekind reals.

  For~\eqref{eq:up5}, $\upp{\L[P]}$ is the set of all $y\in \Q$ such that there is an $x<y$ with $x\notin \L[P]$, i.e.\ such that $x\ge 0$ and $\neg P$.
  Such an $x$ exists if and only if $y>0$ and $\neg P$, which is to say $y\in U[\neg P]$.

  For~\eqref{eq:up6}, $\low{\U[P]}$ is the set of all $x\in \Q$ such that there is a $y>x$ with $y\notin \U[P]$, i.e.\ such that $\neg (y>0 \land P)$.
  Since inequality of rational numbers satisfies trichotomy, this implies either $y>0$ (in which case $\neg P$), or $y\le 0$ (in which case $x<0$); hence $x<0 \lor \neg P$, i.e.\ $x\in L_{\neg P}$.
  Conversely, if $x<0 \lor \neg P$, then in the first case we can find a $y$ with $x<y<0$ and hence $y\notin \U[P]$, while in the second case any $y>x$ satisfies $y\notin \U[P]$.
  Thus, $\low{\U[P]} = L[\neg P]$.

  For~\eqref{eq:up8}, to say $\U[P] < \U[Q]$ means there is an $x\in \U[P]$ with $x\notin \U[Q]$, which means $P\land \neg Q$.
  Similarly, $\L[Q] < \L[P]$ means there is an $x\in \L[P]$ with $x\notin \L[Q]$.
  The first means $x<0$ or $P$, but if $x<0$ then $x\in \L[Q]$; hence $P$, and $x\notin \L[Q]$ implies $\neg Q$.
  The converse is evident.

  For~\eqref{eq:up9}, $\U[P] \le \U[Q] + \U[R]$ means that if $x\in \U[Q]$ and $y\in \U[R]$ then $x+y\in\U[P]$.
  But having $x\in \U[Q]$ and $y\in \U[R]$ mean that $Q$ and $R$ hold, while having $x+y\in\U[P]$ means that $P$ holds, so this is equivalent to $Q\land R \implies P$.

  For~\eqref{eq:up10}, $\L[Q] \le \L[P] + \L[R]$ means that if $z\in \L[Q]$, then there are $x\in \L[P]$ and $y\in \L[R]$ with $x+y=z$.
  Having $z\in \L[Q]$ means either $z<0$ or $Q$.
  If $z<0$, then it can always be written as $x+y$ with $x<0$ and $y<0$, hence $x\in \L[P]$ and $y\in \L[R]$, so the content is in the case when $z\ge 0$ and $Q$.
  But if $x+y\ge 0$, then either $x$ or $y$ must be $\ge 0$, so that $x\in \L[P]$ and $y\in \L[R]$ imply that either $P$ or $R$ must hold.
  Conversely, if $P$ holds we can take $x=z$ and $y=0$, and dually if $R$ holds.
  Thus, $\L[Q] \le \L[P] + \L[R]$ is equivalent to $Q\implies P\lor R$.

  The statement about MacNeille reals follows immediately from~\eqref{eq:up5} and~\eqref{eq:up6}.
  Finally, to say that $\U[P]$ is Dedekind means in particular that either $1\in \U[P]$ (in which case $P$) or $0\in \low{\U[P]} = \L[\neg P]$ (in which case $\neg P$), so that $P\lor\neg P$.
  The converse is immediate, and the case of $\L[P]$ is analogous.
\end{proof}

Thus, any truth value can be represented by an equality or inequality of upper or lower reals, while it is exactly the \nn-stable truth values that can be represented by an equality of MacNeille reals.
(The converse direction of this follows from the observation above that equality in $\Rem$ is \nn-stable.)
However, it is less clear which truth values can be represented by an inequality of MacNeille or Dedekind reals, although we can say that every complemented truth value can be represented by an equality of Dedekind reals.

(In fact, it is known that we have $\Rem = \Red$ if and only if the ``weak law of excluded middle'' holds, i.e.\ $\neg\neg P \lor \neg P$ for all $P$.
This is equivalent to ``de Morgan's law'', that $\neg (P\land Q) \iff \neg P \lor \neg Q$ for all $P,Q$.)


\section{Metric, gauge, and prometric spaces}
\label{sec:metric}

In this section we let $\Re$ stand for one of the sets of extended real numbers --- upper, lower, MacNeille, or Dedekind --- and we write $\Rp = \setof{x\in \Re | x \ge 0 }$.
Note that $\Rp$ is again an ordered commutative monoid.
Whenever a definition involves $\Rp$, it is actually four definitions, one for each kind of real number; if we want to specify which, we add one of the adjectives ``upper'', ``lower'', ``MacNeille'', or ``Dedekind''.

\begin{defn}
  An \textbf{extended quasi-pseudo-metric space} is a set $X$ together with a function $d:X\times X \to \Rp$ such that
  \begin{enumerate}
  \item $d(x,x)=0$ for all $x\in X$ (reflexivity), and
  \item $d(x,z)\le d(x,y)+d(y,z)$ for all $x,y,z\in X$ (transitivity, or the triangle inequality).
  \end{enumerate}
  Such a space is\dots
  \begin{enumerate}[resume]
  \item a \textbf{quasi-pseudo-metric space} if $d(x,y)<\infty$ for all $x,y\in X$.
  \item a \textbf{pseudo-metric space} if $d(x,y)=d(y,x)$ for all $x,y\in X$ (symmetry).
  \item a \textbf{metric space} if $d(x,y)=0$ then $x=y$ (separation).
  \end{enumerate}
\end{defn}

Most of the above terminology is so well-established that we have not attempted to depart from it, but despite their derogatory-sounding name it is really the extended quasi-pseudo-metric spaces that are the fundamental notion.
The additional requirements imposed on a pseudo-metric or metric space are really \emph{separation axioms}, and experience in general topology has shown that a better category is obtained by only imposing separation axioms when necessary rather than including them in definitions.
And allowing infinite distances is arguably of little import topologically.
Moreover, Lawvere~\cite{lawvere:metric-spaces} exhibited extended quasi-pseudo-metric spaces as categories enriched over $(\Rp,\ge,+)$, placing them in a wider context.\fxnote{Should we mention the enriched-multicategory approach to other topological structures?}

In fact, most topological constructions on metric spaces can easily be generalized to the following structures.

\begin{defn}
  An \textbf{(extended) (quasi-)gauge space} is a set $X$ equipped with a set (called a \textbf{gauge}) whose elements are (extended) (quasi-)pseudo-metrics on $X$ (called \textbf{gauging distances}) such that:
  \begin{enumerate}
  \item there exists some gauging distance (nullary filtration), and
  \item if $d_1$ and $d_2$ are gauging distances, there is a gauging distance $d_3$ such that $d_1(x,y)\le d_3(x,y)$ and $d_2(x,y)\le d_3(x,y)$ for all $x,y\in X$ (binary filtration).
  \end{enumerate}
  Such a space is\dots
  \begin{enumerate}[resume]
  \item $\mathbf{T_1}$ if for any $x,y\in X$, if $d(x,y)=0$ for all gauging distances $d$, then $x=y$.
  \item $\mathbf{T_0}$ if for any $x,y\in X$, if $d(x,y)=0$ and $d(y,x)=0$ for all gauging distances $d$, then $x=y$.
  \end{enumerate}
\end{defn}

In other words, a gauge is a filterbase in the poset of (extended quasi-)pseudo-metrics.
Note that the qualifiers ``extended'' and ``quasi-'' carry over from the metrics to the gauge, but all the metrics in a gauge space are automatically assumed to be pseudo; separatedness is named as additional properties that involve all the gauging distances at once.
Evidently the separation axioms $T_0$ and $T_1$ coincide in the symmetric case (gauge spaces), in which case they also imply the stronger topological separation axiom $T_2$ (Hausdorff), as well as $T_3$ (regularity).
Thus, such a space is commonly called simply \textbf{separated}; but in the non-symmetric case it seems worth distinguishing different kinds of separation.

Of course, any pseudo-metric space (of any type) defines a gauge space (of the same type) with exactly one gauging distance; the essential difference between gauge spaces and metric spaces is that gauge spaces need not be ``countable'' in most of the senses that metric spaces must.
Yet more general than gauge spaces are prometric spaces~\cite{cht:one-setting}:

\begin{defn}
  A \textbf{prometric space} is a set $X$ equipped with a set of functions $d:X\times X\to\Rp$, called \textbf{distances}, such that:
  \begin{enumerate}
  \item there exists a distance (nullary filtration),
  \item if $d_1$ and $d_2$ are distances, there is a distance $d_3$ such that $d_1(x,y)\le d_3(x,y)$ and $d_2(x,y)\le d_3(x,y)$ for all $x,y\in X$ (binary filtration),
  \item $d(x,x)=0$ for every distance and all $x\in X$ (reflexivity),
  \item for any distance $d_1$ there is a distance $d_2$ such that for all $x,y,z\in X$ we have $d_1(x,z)\le d_2(x,y)+d_2(y,z)$ (transitivity, or the triangle inequality).
  \end{enumerate}
  A prometric space is\dots
  \begin{enumerate}[resume]
  \item \textbf{symmetric} if for any distance $d_1$ there is a distance $d_2$ such that $d_1(x,y)\le d_2(y,x)$ for all $x,y\in X$.
  \item $\mathbf{T_1}$ if for any $x,y\in X$, if $d(x,y)=0$ for all distances $d$, then $x=y$.
  \item $\mathbf{T_0}$ if for any $x,y\in X$, if $d(x,y)=0$ and $d(y,x)=0$ for all distances $d$, then $x=y$.
  \end{enumerate}
\end{defn}

Since the terminology for prometric spaces is not well-established, we follow~\cite{cht:one-setting} in taking the ``extended quasi'' case to be the default, with symmetry stated explicitly whenever assumed.
Obviously any extended quasi-gauge space (of any type) is also a prometric space (of the same type).

The differences between the various kinds of real numbers, when used as distances, are illustrated by the following example.

\begin{eg}
  Any set $X$ admits a ``discrete'' \emph{upper} prometric, with one distance defined by
  \( d(x,y) = \U[x=y] \).
  Thus $d(x,y)=0$ if and only if $x=y$.
  Note that the triangle inequality follows from~\eqref{eq:up9}.
  This distance is MacNeille if and only if $X$ has \nn-stable equality (i.e.\ $\neg\neg(x=y)\implies (x=y)$), and Dedekind if and only if $X$ has decidable equality (i.e.\ $(x=y)\lor \neg(x=y)$).
  It is always symmetric and $T_1$.

  On the other hand, if $X$ is equipped with an apartness relation $\apart$ (see \cref{sec:point-point}), then it has an ``anti-discrete'' \emph{lower} prometric, with one distance defined by
  \( d(x,y) = \L[x\apart y] \).
  Then $d(x,y)>0$ if and only if $x\apart y$, while $d(x,y)=0$ if and only if $\neg(x\apart y)$.
  The triangle inequality follows from~\eqref{eq:up10}.
  This prometric is always symmetric, and it is $T_1$ if and only if $\apart$ is tight.
\end{eg}

A metric space, and even a prometric space, contains a lot of information.
Some of that information is \emph{numerical}, since points are compared by real number distances.
However, for many purposes we are only interested in ``topological'' or ``geometric'' features that are, for instance, invariant with respect to scaling of distances.
Category-theoretically, this means we consider (pro)metric spaces as the objects of various different categories.
The most important types of morphism between prometric spaces are the following.
\fxnote{Include Cauchy?}

\begin{defn}
  Let $X$ and $Y$ be upper prometric spaces (including, of course, MacNeille or Dedekind prometric spaces) and $f:X\to Y$ a function.
  In what follows, $\ep$ and $\de$ are assumed to range over $\Qp$, and $d_X$ and $d_Y$ are assumed to range over distances on $X$ and $Y$ respectively.
  \begin{enumerate}
  \item $f$ is \textbf{monotone} if whenever $d_X(x,y)=0$ for all $d_X$, then also $d_Y(f(x),f(y))=0$ for all $d_Y$.
  \item $f$ is \textbf{continuous} if for all $x\in X$, $d_Y$, and $\ep$ there are $d_X$ and $\de$ such that for all $y\in X$, if $d_X(x,y)<\de$ then $d_Y(f(x),f(y))<\ep$.
  \item $f$ is \textbf{proximally continuous} if for all $A\subseteq X$, $d_Y$, and $\ep$ there are $d_X$ and $\de$ such that for any $y\in X$, if there exists a $z\in A$ with $d_X(z,y)<\de$ then there exists a $w\in A$ with $d_Y(f(w),f(y))<\ep$.
  \item $f$ is \textbf{uniformly continuous} if for all $d_Y$ and $\ep$ there are $d_X$ and $\de$ such that for all $x,y\in X$, if $d_X(x,y)<\de$ then $d_Y(f(x),f(y))<\ep$.
  \item $f$ is an \textbf{approach map} if for all $x\in X$, $d_Y$, and $k,\eta\in \Qp$, there exists a $d_X$ such that for all $y\in X$ we have $\min(d_Y(f(x),f(y)),k) \le d_X(x,y) +\eta$.
  \item $f$ is \textbf{short}, or a \textbf{contraction}, if for all $d_Y$ there is a $d_X$ such that for all $x,y\in X$ we have $d_Y(f(x),f(y)) \le d_X(x,y)$.
  \end{enumerate}
  We write $\PMetOrdu$, $\PMetTopu$, $\PMetProxu$, $\PMetUnifu$, $\PMetAppu$, and $\PMetu$ for the categories of upper prometric spaces equipped with these five kinds of morphism, respectively.
  If we restrict the objects to be MacNeille or Dedekind prometric spaces, we write $\PMet_m$, $\PMet_d$, and so on.
  If we instead restrict them to be quasi-gauge or metric spaces, we write $\QGau_u$, $\Met_u$, and so on.
\end{defn}

\begin{thm}
  The above kinds of morphism are arranged in increasing order of strictness.
  That is, we have
  \begin{center}
    short $\implies$ approach $\implies$ uniform $\implies$ proximal $\implies$ continuous $\implies$ monotone.
  \end{center}
\end{thm}
\begin{proof}
  It is obvious that short $\implies$ approach: use the same $d_x$.

  For approach $\implies$ uniform, given $d_Y$ and $\ep$, let $\eta = \hfep$ and $k>\ep$.
  Then by the approach condition, we have a $d_X$ such that $\min(d_Y(f(x),f(y)),k) \le d_X(x,y) +\eta$ for all $y$.
  Let $\de=\hfep$; then if $d_X(x,y)<\de$ we have $\min(d_Y(f(x),f(y)),k) < \ep$.
  But $k>\ep$, so this can only happen if $d_Y(f(x),f(y))<\ep$.

  For uniform $\implies$ proximal, obtain $d_X$ and $\de$ from the uniform condition; then we can take $w=z$.
  For proximal $\implies$ continuous, let $A = \singleton{x}$.
  Finally, for continuous $\implies$ monotone, to show $d_Y(f(x),f(y))=0$ it suffices to show $d_Y(f(x),f(y))<\ep$ for all $\ep$, which we can do by finding a $d_X$ and $\de$ and using the fact that $d_X(x,y)=0 <\de$.
\end{proof}

For \emph{lower} prometric spaces, we can use the same definitions of short and approach maps, but the other three should be modified as follows:

\begin{defn}
  Let $X$ and $Y$ be lower prometric spaces and $f:X\to Y$ a function.
  \begin{enumerate}
  \item $f$ is \textbf{monotone} if for all $x,y\in X$ and $d_Y$, if $d_Y(x,y)>0$ there exists a $d_X$ such that $d_X(x,y)>0$.
  \item $f$ is \textbf{continuous} if for all $x\in X$, $d_Y$, and $\ep$ there are $d_X$ and $\de$ such that for all $y\in X$, if $d_Y(f(x),f(y))>\ep$ then $d_X(x,y)>\de$.
  \item $f$ is \textbf{proximally continuous} if for all $A\subseteq X$, $d_Y$, and $\ep$ there are $d_X$ and $\de$ such that for any $y\in X$, if $d_Y(f(w),f(y))>\ep$ for all $w\in A$, then $d_X(z,y)>\de$ for all $z\in A$.
  \item $f$ is \textbf{uniformly continuous} if for all $d_Y$ and $\ep$ there are $d_X$ and $\de$ such that for all $x,y\in X$, if $d_Y(f(x),f(y))>\ep$ then $d_X(x,y)>\de$.
  \end{enumerate}
  We have corresponding categories $\PMet_\ell$, etc.
\end{defn}

\begin{thm}
  For lower prometric spaces we also have
  \begin{center}
    short $\implies$ approach $\implies$ uniform $\implies$ proximal $\implies$ continuous $\implies$ monotone.
  \end{center}
\end{thm}
\begin{proof}
  All the proofs are essentially the same; perhaps the least obvious one is approach $\implies$ uniform.
  Here, once again from $d_Y$ and $\ep$ we get a $d_X$ such that $\min(d_Y(f(x),f(y)),k) \le d_X(x,y) +\hfep$, where $k>\ep$, and we take $\de=\hfep$.
  Then if $d_Y(f(x),f(y))>\ep$, we have $d_X(x,y)+\hfep >\ep$, which (even though we cannot in general subtract lower real numbers) implies $d_X(x,y)>\hfep = \de$.
\end{proof}

Finally, since MacNeille and Dedekind reals can be regarded as both lower or upper reals, we ought to verify that these definitions are consistent.
In fact they are only partially consistent:

\begin{thm}\label{thm:anti-continuous}
  Let $X$ and $Y$ be MacNeille prometric spaces and $f:X\to Y$ a function.
  \begin{enumerate}
  \item $f$ is continuous or uniformly continuous in the upper sense if and only if it is so in the lower sense.
  \item If $f$ is proximally continuous in the upper sense, it is also so in the lower sense.
  \item If $f$ is monotone in the lower sense, it is also so in the upper sense.
  \end{enumerate}
\end{thm}
\begin{proof}
  If $f$ is continuous in the upper sense, then given $x$, $d_Y$ and $\ep$, let $d_X$ and $\de$ be as in upper continuity.
  Then whenever $d_Y(f(x),f(y)) > \ep$, we cannot have $d_X(x,y) < \de$, so by the MacNeille condition we must have $d_X(x,y) > \frac{\de}{2}$.

  Conversely, if $f$ is continuous in the lower sense, then given $x$, $d_Y$ and $\ep$, let $d_X$ and $\de$ be such that $d_Y(f(x),f(y))>\hfep \implies d_X(x,y)>\de$.
  Then if $d_X(x,y)<\de$, we cannot have $d_Y(f(x),f(y))>\hfep$, so by the MacNeille condition we must have $d_Y(f(x),f(y))<\ep$.

  The proofs in the uniform case are essentially identical.

  In the proximal case, given $A$, $d_Y$, and $\ep$, let $d_X$ and $\de$ be as in upper proximal continuity.
  If $d_Y(f(z),f(y))>\ep$ for all $z\in A$, then by contraposition we have $\neg (d_X(w,y)<\de)$ for all $w\in A$, hence $d_X(w,y)>\frac{\de}{2}$.

  Finally, the monotone case follows from the fact that $a\le b$, for MacNeille reals, is equivalent to $\neg(b<a)$, and $\neg\exists$ is equivalent to $\forall\neg$.
\end{proof}

If necessary to disambiguate, we will refer to monotonicity or proximal continuity in the lower sense as \textbf{anti-monotonicity} and \textbf{proximal anti-continuity} respectively.

TODO: Mention that a lower prometric induces an upper prometric by $\upp{d}(x,y) = \upp{d(x,y)}$, but not oppositely, since $\upp{(-)}$ is lax monoidal (for the reversed order $\ge$) but $\low{(-)}$ is colax monoidal.
How functorial is this?


\section{Categories of prometric spaces}
\label{sec:sub-promet}

In this section most of what we say will be independent of the type of real number, so we will omit the subscripts.

Roughly speaking, the goal of general topology is to ``remove'' the undesired information contained in a structure such as a metric, gauge, or prometric space, obtaining more abstract notion of space such as topological, uniform, proximity, and approach spaces.
One reason that is often given for this is the poor behavior of categories of metric spaces: while finite products of metric spaces are easy to construct, countably infinite ones can be constructed only in an \textit{ad hoc} way (and yield categorical products in $\MetTop$ but not in $\Met$), while for uncountably infinite products we are out of luck completely.
However, this problem is already resolved by passage to gauge or prometric spaces:

\begin{thm}
  The category $\PMet$ is topological over $\Set$, i.e.\ its forgetful functor to $\Set$ has initial lifts for all sources and (hence) final lifts for all sinks.
  In particular, it is complete and cocomplete.
  Moreover, the inclusions of $\PMet$ into $\PMetOrd$, $\PMetTop$, $\PMetProx$, $\PMetUnif$, and $\PMetApp$ preserve initial lifts of sources, and its subcategory $\QGau$ is closed under initial lifts of sources; thus these latter categories are also topological,
\end{thm}
\begin{proof}
  Given a family of functions $\{f_i : X \to Y_i\}$, where each $Y_i$ is a prometric space, we define a prometric structure on $X$ consisting of the distances $\max(d^X_1,\dots,d^X_n)$, where each $d^X_j$ is defined by $d^X_j(x,y) = d_j^Y(f_{i_j}(x),f_{i_j}(y))$ for some $i_j$ and some distance $d_j^Y$ on $Y_{i_j}$.
  These distances satisfy filtration by construction; note that the case $n=0$ gives the distance $d(x,y)=0$ for all $x,y$, so that $X$ satisfies nullary filtration even if there are no $Y_i$'s.
  Reflexivity and transitivity follow from their truth for each $Y_i$, while each function $f_i$ is obviously short.
  Moreover, if each $d_j^Y$ is a pseudo-metric, then so is $\max(d^X_1,\dots,d^X_n)$, so that $X$ is a quasi-gauge space if each $Y_i$ is.

  It remains to show that this structure on $X$ is an initial lift preserved by all the inclusions, which is to say that a function $g:Z\to X$ is short, approach, uniform, proximal, continuous, or monotone as soon as all the composites $g\circ f_i$ are.
  In all cases the proof is essentially the same: given a metric $\max(d^X_1,\dots,d^X_n)$ on $X$ and possibly some other things, each metric $d_j^Y$ on $Y_{i_j}$ induces by the appropriate property of $g\circ f_i$ a metric $d^Z_j$ on $Z$.
  Let $d^Z$ dominate all the $d^Z_j$ by filtration of $Z$; then any assumed smallness of $d^Z$ implies smallness of each $d^Z_j$ and hence of each $d_j^Y$, hence also of $\max(d^X_1,\dots,d^X_n)$.
  (In the lower case, the argument runs in reverse, with largeness of $\max(d^X_1,\dots,d^X_n)$ implying that of some $d_j^Y$, hence of some $d^Z_j$, and thus of $d^Z$.)
\end{proof}

Note that if each $Y_i$ is a metric space \emph{and} there are finitely many of them, then $X$ has a maximal metric and hence is isomorphic in $\PMet$ to a metric space; but in general $\Met$ is not closed under initial lifts.

Why, then, do we seek more abstract notions of space?
One answer is that a prometric space contains far \emph{more} information than we are interested in for topology, uniformity, and so on.
From a rigid category-theoretic perspective, one might argue that this removal of information is already accomplished by the definitions of weaker kinds of maps.
However, the fact that it is nevertheless present in the ``internal structure'' of a prometric space can interfere with our intuition and make abstraction and generalization more difficult.
At the technical level, it means that if we want to treat prometric spaces ``topologically'' (for instance), we have the obligation to verify at every step that our constructions respect (iso)morphisms in $\PMetTop$.

Formally, the problem is that $\PMet$ and its supercategories are not \emph{amnestic}, meaning that their fibers over a given set $X$ are only preorders and not partial orders.
In other words, two very different-looking prometrics on the same space might nevertheless be equivalent, i.e.\ isomorphic by the identity function.

In the case of $\PMet$ itself, this can be remedied quite easily: we simply require the prometric to be an ideal rather than an ideal base.
That is, we stipulate that if $d'$ is a distance and $d\le d'$, then $d$ is also a distance.
Call a prometric space \emph{saturated} if it satisfies this condition.
Then if $X$ is saturated, then the condition for $f:X\to Y$ to be short can be restated as
\begin{quote}
  ($\star$) For any distance $d_Y$ on $Y$, the function $d_X:X\times X\to \Re$ defined by $d_X(x,y) = d_Y(f(x),f(y))$ is a distance on $X$.
\end{quote}
Thus, if $X$ and $Y$ are both saturated, an isomorphism $X\cong Y$ ensures that they have \emph{exactly} the same distances.
Moreover, any prometric space $X$ is isomorphic in $\PMet$ to a saturated one: let $\sat{X}$ be the same set $X$ with the distances being all functions $d:X\times X\to \Re$ such that there is a distance $d'$ of $X$ with $d\le d'$.
This is a prometric, since if $d\le d'$ with $d'$ a distance of $X$, there is a distance $d''$ of $X$ with $d(x,z) \le d'(x,z) \le d''(x,y)+d''(y,z)$.
And $\sat{X}$ is saturated since if $d\le d' \le d''$ with $d''$ a distance of $X$ then $d\le d''$, while the identity function $X\to \sat{X}$ is an isomorphism in $\PMet$.

In fact this solution can be extended to all the supercategories of $\PMet$ as well.
Note that all the continuity conditions on maps $f:X\to Y$ between prometric spaces can be rephrased to be of the form ``for any $d_Y$, the function defined by $d_X(x,y) = d_Y(f(x),f(y))$ has some property''.
Thus, it is natural to simply demand a stronger form of saturation, that any function with the relevant property is a distance on $X$.

This works literally for quasi-gauge spaces if instead of ``any function'' we say ``any pseudo-metric'', but not for prometrics.
The problem is that for an arbitrary function $d:X\times X \to \Re$ satisfying the ``some property'', there may not be another such function $d'$ to participate with $d$ in the triangle inequality.
Instead we take an impredicative approach.

Note that the collection of prometrics on a set $X$ is partially ordered by inclusion.
Moreover, it is a complete lattice, with suprema constructed by closing up unions under finite maxima.
That is, if $\{\cM_i\}$ is a set of prometrics on $X$, then $\bigvee_i \cM_i$ consists of all functions of the form $\max(d_1,\dots,d_n)$, where each $d_j \in \cM_{i_j}$ for some $i_j$.

\begin{defn}
  %, and let $d$ range over functions $X\times X\to \Re$.
  % \begin{enumerate}
  % \item
  A prometric $\cM$ on a set $X$ is \textbf{saturated} (resp.\ \textbf{approach}, \textbf{uniform}, \textbf{proximal}, \textbf{topological}, \textbf{ordered}) if whenever $\cM'$ is a prometric on $X$ such that $\id_X:(X,\cM) \to (X,\cM')$ is short (resp.\ approach, uniformly continuous, proximally continuous, continuous, monotone) we have $\cM'\subseteq \cM$.
  % \item $\cM$ is \textbf{approach} if $d$ is a distance as soon as for all $x\in X$ and $k,\eta\in \Qp$, there is a distance $d'$ such that for all $y\in X$ we have $\min(d(x,y),k) \le d'(x,y) +\eta$.
  % \item $\cM$ is \textbf{uniform} if $d$ is a distance as soon as for all $\ep$ there is a distance $d'$ and a $\de$ such that for all $x,y\in X$, if $d'(x,y)<\de$ then $d(x,y)<\ep$.
  % \item $\cM$ is \textbf{proximity} if $d$ is a distance as soon as for all $A\subseteq X$ and $\ep$ there is a distance $d'$ and a $\de$ such that for any $y\in X$, if there exists a $z\in A$ with $d'(z,y)<\de$ then there exists a $w\in A$ with $d(w,y)<\ep$.
  % \item $\cM$ is \textbf{topological} if $d$ is a distance as soon as for all $x\in X$ and $\ep$ there is a distance $d'$ and a $\de$ such that for all $y\in X$, if $d'(x,y)<\de$ then $d(x,y)<\ep$.
  % \item $\cM$ is \textbf{ordered} if $d$ is a distance as soon as $d(x,y)=0$ whenever $d'(x,y)=0$ for all distances $d'$.
  % \end{enumerate}
  %(For lower prometrics, we modify the last four definitions in the obvious way.)
  We write $\SatPMet$, $\AppPMet$, $\UnifPMet$, $\ProxPMet$, $\TopPMet$, and $\OrdPMet$ for the categories whose objects are those defined above and whose morphisms \emph{in all cases} are the short maps.
\end{defn}

Note that this formal definition of saturation agrees with the simpler one given formerly.
Since plain saturation is weaker than all the above notions, it follows that the morphisms in each of these categories can again be characterized by the more straightforward property ($\star$).
Thus each of these categories is amnestic over $\Set$.
Moreover, we have fully faithful inclusions
\begin{multline*}
  \OrdPMet \into \TopPMet \into \ProxPMet \into \UnifPMet\\
  \into \AppPMet \into \SatPMet \into \PMet.
\end{multline*}

\begin{thm}
  Let $X$ and $Y$ be prometric spaces and $f:X\to Y$ a function.
  \begin{enumerate}
  \item If $X$ is approach, then $f$ is short if and only if it is an approach map.
  \item If $X$ is uniform, then $f$ is short if and only if it is uniformly continuous.
  \item If $X$ is proximity, then $f$ is short if and only if it is proximally continuous.
  \item If $X$ is topological, then $f$ is short if and only if it is continuous.
  \item If $X$ is ordered, then $f$ is short if and only if it is monotone.
  \end{enumerate}
\end{thm}
\begin{proof}
  In all cases ``only if'' is always true, so it suffices to show ``if''.
  Let $\cM$ be the given prometric on $X$, and let $\cM_f$ be the initial prometric on $X$ induced by $f$.
  Then $f:(X,\cM_f) \to Y$ is short, while $f:(X,\cM) \to Y$ is short, approach, uniformly continuous, proximally continuous, continuous, or monotone if and only if $\id_X:(X,\cM) \to (X,\cM_f)$ is so.
  But if $\id_X$ has some form of continuity, then the appropriate kind of saturation for $\cM$ implies that $\cM_f \subseteq \cM$, and hence $\id_X$ is also short.
\end{proof}

\begin{thm}
  Each of the categories $\SatPMet$, $\AppPMet$, $\UnifPMet$, $\ProxPMet$, $\TopPMet$, and $\OrdPMet$ is coreflective in $\PMet$, and in the case of $\SatPMet$ the adjunction is an equivalence.
\end{thm}
\begin{proof}
  Let $\sC$ be one of the above categories; we will also speak of ``$\sC$-continuous maps'' (e.g.\ uniformly continuous, monotone, etc.)
  Let the coreflection of $(X,\cM)$ be the join, in the poset of prometrics on $X$, of all prometrics $\cM'$ such that $\id_X :(X,\cM) \to (X,\cM')$ is $\sC$-continuous; call it $\sat{\cM}$.
  Then $\id_X :(X,\cM) \to (X,\sat{\cM})$ is also $\sC$-continuous, while $\id_X : (X,\sat{\cM}) \to (X,\cM)$ is short since $\cM \subseteq \sat{\cM}$.

  (In particular, $\id_X$ is an isomorphism in the corresponding \emph{supercategory} of $\PMet$ --- that is, $\PMetTop$, $\PMetUnif$, etc.
  In the case $\sC=\SatPMet$ this shows that the inclusion is an equivalence.)

  Now if $f:Y\to (X,\cM)$ is short and $Y\in\sC$, then $f:Y\to (X,\cM)$ is also $\sC$-continuous, and hence so is $f:Y\to (X,\sat{\cM})$.
  But $Y\in\sC$, so $f:Y\to (X,\sat{\cM})$ is also short.
  Thus it is the unique factorization of $f:Y\to (X,\cM)$ through $\id_X : (X,\sat{\cM}) \to (X,\cM)$.
\end{proof}

\begin{cor}
  The following composites are equivalences of categories:
  \begin{gather*}
    \OrdPMet \into \PMet \into \PMetOrd\\
    \TopPMet \into \PMet \into \PMetTop\\
    \ProxPMet \into \PMet \into \PMetProx\\
    \UnifPMet \into \PMet \into \PMetUnif\\
    \AppPMet \into \PMet \into \PMetApp
  \end{gather*}
  In particular, the functors $\PMet \to \PMetTop$ and so on all have fully faithful left adjoints.\qed
\end{cor}

A constructivist, therefore, may reasonably argue at this point that we have perfectly serviceable notions of topological space, uniform space, and so on.
Indeed, we have four of each of them, depending on what kind of real numbers we take as distances.
However, there are still reasons to be unsatisfied.

For instance, a topological prometric is an \emph{immense} object, containing all sorts of strange-looking distances; it is hard to get any intuitive feel for such a thing.
It is much easier to understand and work with a less ``flabby'' structure.
Moreover, such leaner structures are more amenable to generalization, notably to the point-free context.
For this reason, we move on to consider structures defined \emph{relations} between points or subsets rather than distances between them.


\section{Point-point relations: order and apartness}
\label{sec:point-point}
\label{sec:order}

We begin with the simplest, almost trivial, case, of a single relation between pairs of points.
There are two important such ``global'' relations in a prometric space.
As always, $d$ will range over distances in a prometric space $X$.

\begin{defn}\label{def:promet-ord}
  Let $x,y\in X$ be points of a prometric space.
  \begin{itemize}
  \item We say $x\leapx y$ if $\forall d, d(x,y)=0$.\fxwarning{ordering convention?}
  \item We say $x\oapt y$ if $\exists d, d(x,y)>0$.
  \end{itemize}
\end{defn}

Since $\neg(a<b)$ is equivalent to $b\le a$ for located real numbers, in a prometric space we have
\begin{equation}
  (x\leapx y) \iff \neg(x\oapt y).\label{eq:leapx-oapt}
\end{equation}
However, $\oapt$ cannot constructively be expressed in terms of $\leapx$; to say $x\oapt y$ is stronger than $x\neq y$ or even $x\not\leapx y$.

The abstract structure possessed by $\leapx$ is obvious and well-known.

\begin{thm}\label{thm:pmet-preord}
  Let $X$ be a prometric space.
  \begin{enumerate}
  \item The relation $\leapx$ is a \textbf{preorder} (i.e.\ a reflexive transitive relation).
  \item If $X$ is symmetric, then $\leapx$ is a symmetric relation, and hence an \textbf{equivalence relation} (which we will usually write as $\approx$).
  \item $X$ is $T_0$ if and only if $\leapx$ is antisymmetric ($x\leapx y$ and $y\leapx x$ together imply $x=y$), hence a \textbf{partial order}.
  \item $X$ is $T_1$ if and only if $\leapx$ coincides with equality.
  \end{enumerate}
\end{thm}
\begin{proof}
  Reflexivity and transitivity of $\leapx$ follow from reflexivity and transitivity (i.e.\ the triangle inequality) of a prometric, and likewise for symmetry and the separation axioms.
\end{proof}

Of course, $\leapx$ is an instance of the \emph{specialization order} underlying a topology; we will return to this in \cref{sec:point-point}.

The structure possessed by $\oapt$ is less familiar to classical mathematicians, but well-known to constructive ones.  
For reasons that will become clear later, we first define a weaker notion and then the more familiar stronger one.

\begin{defn}\label{def:anti-preorder}
  A \textbf{anti-preorder} is a binary relation $\oapt$ on a set $X$ such that:
  \begin{enumerate}
  \item $\neg(x\oapt x)$ for all $x$ (irreflexivity).
  \item If $x\oapt z$ and $\neg(y\oapt z)$, then $x\oapt y$ (bilocal transitivity).
  \end{enumerate}
  an anti-preorder is\dots
  \begin{enumerate}[resume]
  \item \textbf{bilocally decomposable}, or a \textbf{comparison}, if whenever $x\oapt z$, then for any $y$ we have $x\oapt y$ or $y\oapt z$.
    (Note that this implies bilocal transitivity.)
  \item \textbf{symmetric} if $x\oapt y$ implies $y\oapt x$ (symmetry).
  \item an \textbf{apartness relation} (usually written $\apart$) if it is a symmetric comparison.
  \item \textbf{connected} if $(\neg(x\oapt y)\land \neg(y\oapt x))\implies x=y$.
  \item \textbf{tight} if $\neg(x\oapt y) \implies x=y$.
  \end{enumerate}
\end{defn}

Tightness is usually defined only for symmetric relations, in which case it of course coincides with connectedness.
But in the non-symmetric case, it seems useful to distinguish the two notions.\fxnote{Is this reasonable?}

\begin{warn}
  Some constructive mathematicians write $\neq$ for an arbitrary apartness relation.
  We will always reserve slashed symbols such as $\neq,\not\leapx,\napprox,\notin$ for logical negations (i.e.\ $x\neq y$ means $\neg(x=y)$, $x\not\leapx y$ means $\neg(x\leapx y)$, etc.), and use other symbols such as $\apart$ and $\oapt$ for stronger ``positive'' relations.
\end{warn}

\begin{thm}
  Let $X$ be a prometric space.
  \begin{enumerate}
  \item The relation $\oapt$ is a comparison anti-preorder.  
  \item If $X$ is symmetric, $\oapt$ is an apartness relation.
  \item $X$ is $T_0$ if and only if $\oapt$ is connected.
  \item $X$ is $T_1$ if and only if $\oapt$ is tight.
  \end{enumerate}
\end{thm}
\begin{proof}
  Irreflexivity of $\oapt $ follows directly from reflexivity of a prometric.
  Comparison is rather trickier: if $d(x,z)>0$, we have another distance $d'$ with $d'(x,y)+d'(y,z)\ge d(x,z)$.
  By locatedness of real numbers, $d'(x,y)$ and $d'(y,z)$ are both either $>0$ or $<\frac12 d(x,z)$.
  Thus either $x\oapt y$ or $y\oapt z$ or $d'(x,y)+d'(y,z) < d(x,z)$; but the latter case is a contradiction.

  Symmetry is obvious, and the last two statements follow from~\eqref{eq:leapx-oapt} and \cref{thm:pmet-preord}.
\end{proof}

In particular, the equality relation of a $T_0$ prometric space is (like that of any set with a connected irreflexive relation) \nn-stable, for $\neg\neg(x=y)$ implies $\neg(x\oapt y)$, hence $x\leapx y$ and thus $x=y$.

Preorders and anti-preorders are closely related.
In fact, in classical mathematics they are entirely equivalent notions.
To state the corresponding constructive fact, we first define the appropriate categories.

\begin{defn}
  Let $f:X\to Y$ be a function.
  \begin{enumerate}
  \item If $X$ and $Y$ are preorders, then $f$ is \textbf{monotone} if $x\leapx y \implies f(x)\leapx f(y)$ for all $x,y\in X$.
    We write $\Preord$ for the category of preorders and monotone maps.
  \item If $X$ and $Y$ are anti-preorders, then $f$ is \textbf{monotone}, or \textbf{strongly extensional}, if $f(x)\oapt f(y) \implies x\oapt y$ for all $x,y\in X$.
    We write $\APreord$ for the category of anti-preorders and monotone maps.
  \end{enumerate}
\end{defn}

\begin{thm}\label{thm:ord-le-apt}
  If $X$ is a preorder, then the relation $\neg(x\leapx y)$ is an anti-preorder $\anti(X)$.
  Dually, if $X$ is an anti-preorder, then the relation $\neg(x\oapt y)$ is a preorder $\neigh(X)$.
  These operations define an idempotent adjunction $\anti\adj\neigh$:
  \[\anti : \Preord \rightleftarrows \APreord : \neigh\]
\end{thm}
\begin{proof}
  Reflexivity is easy in both directions.
  If $X$ is a preorder, then for transitivity of its dual, suppose $\neg(x\leapx z)$ and $\neg\neg(y\leapx z)$; we must show $\neg(x\leapx y)$.
  Suppose $x\leapx y$; then since we are trying to prove a contradiction, we may assume $y\leapx z$, hence $x\leapx z$ by transitivity of $\leapx$, contradiction our other assumption.

  Dually, if $X$ is an anti-preorder, for transitivity of its dual suppose $\neg (x\oapt y)$ and $\neg (y\oapt z)$; we must show $\neg(x\oapt z)$.
  Suppose $x\oapt z$; then by transitivity of $\oapt$ and our second assumption we have $x\oapt y$, contradicting our first assumption.

  Functoriality in both directions is obvious from the contravariance of $\neg$.
  For the unit and counit of the adjunction we take identity functions.
  On one hand, if $X$ is a preorder, then $\neigh(\anti(X))$ is the same set $X$ with the relation $\neg\neg(x\leapx y)$.
  Since $(x\leapx y) \implies \neg\neg(x\leapx y)$, the identity function is a monotone map $X\to \neigh(\anti(X))$.
  On the other hand, if $X$ is an anti-preorder, then $\anti(\neigh(X))$ is the same set $X$ with the relation $\neg\neg(x\oapt y)$.
  Since $(x\oapt y) \implies \neg\neg(x\oapt y)$, the identity function is a monotone map $\anti(\neigh(X)) \to X$.
  The triangle identities are obviously satisfied.

  To say that an adjunction is \emph{idempotent} means that the unit or counit (hence also the other) is an isomorphism when restricted to objects in the image of the right or left adjoint, respectively.
  In our case, this follows from the fact that $\neg P \iff \neg\neg\neg P$ for any truth value $P$.
\end{proof}

\cref{thm:ord-le-apt} gives some partial justification for defining the notion of anti-preorder using bilocal transitivity instead of bilocal decomposability (comparison), since there seems no general way to make a comparison by negating a preorder.
Further justification will have to wait for \cref{thm:top-ord} and \cref{sec:syntopogeny}.

Any idempotent adjunction restricts to an equivalence between the full images of the right and left adjoints.
Thus, preorders that are \textbf{\nn-stable} (i.e.\ $\neg\neg(x\leapx y) \implies x\leapx y$) are equivalent to anti-preorders that are \nn-stable.

\begin{thm}\label{thm:pmet-monotone}
  Let $X$ and $Y$ be prometric spaces and $f:X\to Y$ a continuous function.
  Then $f$ is a monotone map of both preorders and anti-preorders.
\end{thm}
\begin{proof}
  For the first, given any $d_Y$ and $\ep>0$ we have a $d_X$ and $\de>0$ such that $d_X(x,y)<\de$ implies $d_Y(f(x),f(y))<\ep$.
  But $d_X(x,y)<\de$ is always true if $x\leapx y$; hence so is $d_Y(f(x),f(y))<\ep$, and thus $f(x)\leapx f(y)$.

  For the second, if $f(x)\oapt f(y)$ we have $d_Y(f(x),f(y))>0$ for some $d_Y$.
  Let $\ep = \frac12 \,d_Y(f(x),f(y))$; then by \cref{thm:cocontinuous} we have a $d_X$ and a $\de>0$ such that if $d_Y(f(x),f(y))>\ep$ (which is true) then $d_X(x,y)>\de$, hence $x\oapt y$.
\end{proof}

In fact,~\eqref{eq:leapx-oapt} means that the preorder of a prometric space is obtained from its anti-preorder via the functor $\neigh$ from \cref{thm:ord-le-apt}, so that the first statement of \cref{thm:pmet-monotone} follows from the second.
This is an instance of the fact that constructively, it is often better to work with anti-preorders (such as apartness relations) than preorders (such as equivalence relations), since many preorders are the dual of some anti-preorder but not conversely.
However, this seems to be due mainly to the impoverished nature of a single binary relation; in later sections we will see that for more refined topological structures we can impose natural conditions (satisfied by any prometric space) to make the ``closeness'' and ``apartness'' notions coincide.

% \begin{thm}\label{thm:ord-topconcrete}
%   The categories of preorders, anti-preorders, and comparison anti-preorders are ``topological over $\Set$'', i.e.\ their forgetful functor to $\Set$ has initial lifts for all sources and (hence) final lifts for all sinks.
%   In particular, they are complete and cocomplete.
% \end{thm}
% \begin{proof}
%   Given a source $\{f_i : X \to Y_i\}$ with each $Y_i$ a preorder, define $x\leapx y$ for $x,y\in X$ to mean that $f_i(x)\leapx_i f_i(y)$ for all $i$.
%   This is evidently a preorder and the initial lift.

%   Similarly, given $\{f_i : X \to Y_i\}$ with each $Y_i$ an anti-preorder, define $x\oapt y$ for $x,y\in X$ to mean that $f_i(x)\oapt_i f_i(y)$ for \emph{some} $i$.
%   Irreflexivity is obvious.
%   For bilocal transitivity, if $x\oapt z$, so that $f_i(x)\oapt_i f_i(z)$ for some $i$, then if $\neg(y\oapt z)$ we have $\neg (f_i(y)\oapt_i f_i(z))$, hence $f_i(x) \oapt_i f_i(y)$ and so $x\oapt y$.
%   A similar argument works for comparison.
% \end{proof}


\section{Point-set relations: topology}
\label{sec:point-set}
\label{sec:topology}

The specialization order and the apartness relation, while important, do not usually capture very much of the topological information in a prometric space.
The crucial step that brings us into the world of general topology is to consider relations involving \emph{sets} as well as points.
There are three fundamental relations between a point and a set that can be defined in a prometric space $X$ (with $d$ ranging over distances and $\ep$ over positive reals, as always):

\begin{defn}\label{defn:toprels}
  Let $X$ be a prometric space, $x\in X$, and $A\subseteq X$.
  \begin{enumerate}
  \item $x\ll A$ if $\exists d, \exists \ep, \forall y, (d(x,y)<\ep \implies y\in A)$ ($A$ is a \textbf{neighborhood} of $x$).
  \item $x\bowtie A$ if $\exists d, \exists \ep, \forall y, (y\in A \implies d(x,y)>\ep)$ ($x$ is \textbf{apart} from $A$).
  \item $x\approx A$ if $\forall d, \forall \ep, \exists y, (y\in A \land d(x,y)<\ep)$ ($x$ is \textbf{close} to $A$).
  \end{enumerate}
\end{defn}

We can now axiomatize the basic properties of these relations.
Our treatment is strongly inspired by that of~\cite{bridges-vita}, but we make some minor choices differently, in order to obtain better category-theoretic properties and also to fit better into the general framework we will articulate later.
In \cref{sec:top-compare} we will compare our definitions to theirs.


\subsection{Neighborhood, apartness, and closure spaces}
\label{sec:top}

In contrast to \cref{sec:point-point}, here we will also consider structures that lack transitivity.
We write $\cpl{A} = X\setminus A = \setof{x\in X | x\notin A}$ for the relative complement of a subset $A\subseteq X$.

\begin{defn}
  A \textbf{pretopological neighborhood space} is a set $X$ with a relation $\ll$ between points and subsets such that
  \begin{enumerate}
  \item If $x\ll A$ and $A\subseteq B$, then $x\ll B$ (isotony).
  \item If $x\ll A$, then $x\in A$ (reflexivity).
  \item $x\ll X$ for all $x\in X$ (nullary additivity).
  \item If $x\ll A$ and $x\ll B$, then $x\ll A\cap B$ (binary additivity).
  \end{enumerate}
  A pretopological neighborhood $X$ space is\dots
  \begin{enumerate}[resume]
  \item \textbf{topological} if whenever $x\ll A$, we have $x \ll \setof{ y | y\ll A}$ (transitivity).
  % \item \textbf{\nn-stable} if whenever $x\ll A$, there is a subset $B$ such that $x\ll \cpl{B} \subseteq A$.
  \item \textbf{locally decomposable} if whenever $x\ll A$, there is a subset $B$ such that $x\ll B$ and $\forall y, (y\notin B \lor y \ll A)$.
  \end{enumerate}
  If $X$ and $Y$ are pretopological neighborhood spaces, a function $f:X\to Y$ is \textbf{continuous} if $f(x)\ll A$ implies $x\ll f\inv(A)$.
  We write $\PTop$ for the category of pretopological neighborhood spaces and continuous maps, and $\Top$
  % $\PTopnn$,
  and $\ldTop$ for its subcategories of topological %, \nn-stable,
  and locally decomposable neighborhood spaces.
\end{defn}

Note that a locally decomposable neighborhood space is automatically topological, % and \nn-stable,
since if $\forall y, (y\notin B \lor y \ll A)$ then % $\neg\neg B \subseteq A$ and also
$B \subseteq \setof{ y | y\ll A}$.
Conversely, in classical mathematics any topological neighborhood space is locally decomposable, since we can take $B = \setof{ y | y\ll A}$.
So the distinction between the two is only visible in the constructive world.

\begin{defn}
  A \textbf{pretopological apartness space} (sometimes a \textbf{point-set apartness space} for emphasis) is a set $X$ with a relation $\bowtie$ between points and subsets such that
  \begin{enumerate}
  \item If $x\bowtie A$ and $B\subseteq A$, then $x\bowtie B$ (isotony).
  \item If $x\bowtie A$, then $x\notin A$ (reflexivity).
  \item $x\bowtie \emptyset$ for all $x\in X$ (nullary additivity).
  \item If $x\bowtie A$ and $x\bowtie B$, then $x\bowtie A\cup B$ (binary additivity).
  \end{enumerate}
  A pretopological apartness space is\dots
  \begin{enumerate}[resume]
  \item \textbf{topological} if whenever $x\bowtie A$, we have $x\bowtie \setof{ y | y \not\bowtie A }$ (transitivity).
  % \item \textbf{\nn-stable} if whenever $x\bowtie A$, we have $x\bowtie \cpl{\cpl{A}}$.
  \item \textbf{locally decomposable} if whenever $x\bowtie A$, there is a subset $B$ such that $x\bowtie B$ and $\forall y, (y\in B \lor y\bowtie A)$.
  \end{enumerate}
  If $X$ and $Y$ are pretopological apartness space, a function $f:X\to Y$ is \textbf{continuous} if $f(x)\bowtie f(A)$ implies $x\bowtie A$.
  We write $\APTop$ for the category of pretopological apartness spaces and continuous maps, and $\ATop$
  % , $\APTopnn$,
  and $\ldATop$ for its subcategories of topological %, \nn-stable,
  and locally decomposable apartness spaces.
\end{defn}

As before, a locally decomposable pretopological apartness space is topological, since if $\forall y, (y\in B \lor y\bowtie A)$ then $\setof{ y | y \not\bowtie A } \subseteq B$.
% Moreover, in slight contrast to the neighborhood case, any topological apartness space is \nn-stable, since $\cpl{\cpl{A}}\subseteq \setof{ y | y \not\bowtie A }$ by the contrapositive of reflexivity.
Similarly, in classical mathematics any topological apartness space is locally decomposable, since we can take $B = \setof{y|y\not\bowtie A}$.

As we will see below, the relation $\approx$ is not very useful to axiomatize constructively, so we will not bother to isolate its constructive variants.
We will however name the version that omits binary additivity, since that axiom is usually impossible to achieve constructively.

\begin{defn}
  A \textbf{Moore closure space} is a set $X$ with a relation $\approx$ between points and subsets such that
  \begin{enumerate}
  \item If $x\approx A$ and $A\subseteq B$, then $x\approx B$ (isotony).
  \item If $x\in A$, then $x\approx A$ (reflexivity).
  \item $x\napprox \emptyset$ for all $x\in X$ (nullary additivity).
  \item If $x\approx \setof{ y | y\approx A}$, then $x\approx A$ (transitivity).
  \end{enumerate}
  A Moore closure space is \textbf{topological} if it also satisfies
  \begin{enumerate}[resume]
  \item If $x\approx A\cup B$, then $x\approx A$ or $x\approx B$ (binary additivity).
  \end{enumerate}
  If $X$ and $Y$ are topological or Moore closure spaces, a function $f:X\to Y$ is \textbf{continuous} if $x\approx A$ implies $f(x) \approx f(A)$.
\end{defn}

A relation between points and subsets can equivalently be defined in terms of an operation taking each subset to the subset of points related to it.
For instance, in a topological neighborhood space, the \textbf{interior} of a set $A$ is
\[ \int(A) = \setof{ x | x\ll A } \]
The axioms of $\ll$ say exactly that $\int(-)$ is a left-exact comonad on the powerset of $X$.
Hence it is determined by its fixed points, which are called \textbf{open sets}, and are closed under arbitrary unions and finite intersections.
This is, of course, the usual definition of a \textbf{topological space}.
We will stick to working directly with $\ll$, however, since it is more analogous to $\bowtie$ and generalizes better.
Our ``pretopological neighborhood spaces'' also coincide with the usual classical definition of \emph{pretopological space}, usually defined as a set equipped with a filter of neighborhoods at each point.

Similarly, in a Moore closure space, the \textbf{closure} of a set $A$ is
\[ \cl(A) = \setof{x | x \approx A } \]
The axioms of $\approx$ say exactly that $\cl(-)$ is a comonad on the powerset of $X$, which is right-exact exactly when $X$ is topological.
Hence it is determined by its fixed points, which are called \textbf{closed sets}, and are closed under arbitrary intersections (and finite unions, if $X$ is topological).

Finally, in a pretopological apartness space, the \textbf{exterior} of a set $A$ is
\[ \ext(A) = \setof{ x | x\bowtie A } \]
It is less natural to rephrase the axioms of $\bowtie$ in terms of $\ext$, however.

\begin{thm}\label{thm:pmet-top}
  If $X$ is a prometric space, then the relations $\ll$, $\bowtie$, and $\approx$ from \cref{defn:toprels} make it into a locally decomposable topological neighborhood space, a locally decomposable topological apartness space, and a Moore closure space, plus a topological closure space if excluded middle holds.
\end{thm}
\begin{proof}
  In all cases, isotony and nullary additivity are obvious, while reflexivity follows directly from prometric reflexivity ($d(x,x)=0$).
  Binary additivity for $\ll$ and $\bowtie$ follow directly from the definitions of $\cap$ and $\cup$.

  Binary additivity for $\approx$, however, requires that if $\forall d,\forall \ep, \exists y\in A\cup B, d(x,y)<\ep$, then either $\forall d,\forall \ep, \exists y\in A, d(x,y)<\ep$ or $\forall d,\forall \ep, \exists y\in B, d(x,y)<\ep$.
  If excluded middle holds, we can prove this by contradiction: if neither of the two desired conclusions holds, then there are $d_1,\ep_1$ such that $d_1(x,y)\ge \ep_1$ for all $y\in A$, and also $d_2,\ep_2$ such that $d_2(x,y)\ge\ep_2$ for all $y\in B$.
  Then by binary filtration, there is a $d_3$ such that $d_1\le d_3$ and $d_2\le d_3$, and by assumption there is a $y\in A\cup B$ such that $d_3(x,y)<\min(\ep_1,\ep_2)$.
  But our assumptions show that this $y$ cannot be in $A$ or in $B$, a contradiction.
  Constructively, however, it seems impossible to prove this axiom.

  For local decomposability of $\ll$, if $x\ll A$ then we have a $d,\ep$ with $d(x,y)<\ep \implies y\in A$.
  Let $d'$ be as in prometric transitivity for $d$, and $B = \setof{ y | d'(x,y) < \fep13}$.
  Then certainly $x\ll B$, while for any $y$ we have either $d'(x,y)<\fep23$ or $d'(x,y)>\fep13$.
  In the first case, if $d'(x,y)<\fep23$, then for any $z$ with $d'(y,z)<\fep13$ we have $d(x,z) \le d'(x,y)+d'(y,z) <\ep$, so $z\in A$; thus $y\ll A$.
  In the second case, if $d'(x,y)>\fep13$ then certainly $y\notin B$.
  Local decomposability of $\bowtie$, and transitivity of $\approx$, are similar.
\end{proof}

In classical mathematics, all three notions of topological space are equivalent, via the following definitions:
\[
\begin{array}{ccccc}
  x\ll A & \iff & x \bowtie \cpl{A} & \iff & x\napprox \cpl{A}\\
  x\bowtie A & \iff & x\napprox A & \iff & x \ll \cpl{A}\\
  x\approx A & \iff & x \not\ll \cpl{A} & \iff & x \not\bowtie A
\end{array}
\]
Constructively, this is no longer so, but we do have various relationships between the three structures.
Closure spaces are unsatisfactory as a \emph{basic} notion of space because their binary additivity axiom cannot be proven even for metric spaces, but the following notions are nevertheless useful:

\begin{thm}\label{thm:top-closed}
  Let $X$ be a topological neighborhood space.
  \begin{enumerate}
  \item If we define $x\approx A$ to mean
    \[\forall B, (x\ll B \implies A\cap B \text{ is inhabited}),\]
    then $X$ becomes a Moore closure space.\label{item:top-ll-wkcl}
    Its closed sets are called \textbf{weakly closed} in $X$.
  \item If we define $x\approx A$ to mean
    \[\forall B, (x\ll B \implies A\cap B \neq \emptyset),\]
    then $X$ becomes a Moore closure space.\label{item:top-ll-strcl}
    Its closed sets are called \textbf{strongly closed} in $X$ (they are the complements of the open sets).
  \end{enumerate}
  Moreover, both these constructions are functorial.
\end{thm}
\begin{proof}
  In both cases, all the axioms are easy except binary additivity and transitivity.
  Binary additivity fails constructively for essentially the same reason that it failed for prometric spaces.

  For transitivity of~\ref{item:top-ll-wkcl}, if $x\approx \setof{y | y\approx A}$, then whenever $x\ll B$, we also have $x\ll \int(B)$ by transitivity for $\ll$.
  Hence there is a $y\in \int(B)$ such that $y\approx A$; but then $y\ll \int(B)$, and so by definition of $\approx$ there is a $z\in\int(B)$ with $z\in A$.
  Since then $z\in B$ too, we have $x\approx A$.

  For transitivity of~\ref{item:top-ll-strcl}, let $x\approx \setof{y | y\approx A}$ and suppose for contraction that $x\ll B$ and $A\cap B =\emptyset$.
  Then also $x\ll \int(B)$ and $A\cap \int(B)=\emptyset$.
  We will show that $\setof{y | y\approx A} \cap \int(B) = \emptyset$, for which purpose suppose it contains an element $y$.
  But then $y\approx A$ and $y\in \int(B)$, hence $y\ll \int(B)$; thus by definition of $\approx$ we have $A\cap \int(B) \neq \emptyset$, a contradiction.

  For functoriality, suppose $X$ and $Y$ are topological neighborhood spaces and $f:X\to Y$ is $\ll$-continuous.
  If $x\approx A$ weakly, suppose given $B\subseteq Y$ with $f(x)\ll B$; then by $\ll$-continuity, we have $x\ll f\inv(B)$.
  Since $x\approx A$, we have $A\cap f\inv(B)$ inhabited, hence $f(A) \cap B$ inhabited.
  Thus, $f(x)\approx f(A)$.
  The proof for strong $\approx$ is essentially the same.
\end{proof}

In~\cite{bridges-vita} the weak closure relation $x\approx A$ is written as $\mathbf{near}(x,A)$, and an example is given showing that binary additivity for this relation in the metric space $\R$ implies the Limited Principle of Omniscience.
We will not say much more about closure spaces.

Topological neighborhood spaces and topological apartness spaces, on the other hand, are both reasonable constructive notions of ``space''.
In support of this we note the following:

\begin{thm}\label{thm:top-topconcrete}
  The categories $\PTop$ and $\APTop$ are ``topological over $\Set$'', i.e.\ their forgetful functor to $\Set$ has initial lifts for all sources and (hence) final lifts for all sinks.
  In particular:
  \begin{enumerate}
  \item They are complete and cocomplete.
  \item Induced and coinduced structures exist.
  \item The forgetful functor to $\Set$ has both a left adjoint (discrete spaces) and a right adjoint (indiscrete spaces).
  \item Every continuous function factors uniquely as a surjection followed by an embedding (an inclusion with the induced structure), and also as a quotient (a surjection with the coinduced structure) followed by an injection.
  \end{enumerate}
  Moreover, their subcategories $\Top$, $\ATop$, %$\PTopnn$,
  $\ldTop$, and $\ldATop$ % (but not $\APTopnn$)
  are closed under initial lifts for sources.
  In particular, they are each also topological over $\Set$, and are reflective in the corresponding supercategory.
\end{thm}
\begin{proof}
  Note that pretopological neighborhood and apartness relations are closed under directed unions of point-set relations.
  Directedness is necessary for preservation of additivity.
  However, any family of such relations still has a join in the poset of such relations, constructed explicitly by closing up under finite additivity.
  Moreover, such joins preserve transitivity %, \nn-stability in the neighborhood case,
  and local decomposability.

  Explicitly, given a family $\{\ll_i\}$ of pretopological neighborhood relations, define $x\ll A$ to mean that there exists a finite family $i_1,\dots,i_n$ and sets $B_1,\dots, B_n$ such that $\bigcap_{j=1}^n B_j \subseteq A$ and $x\ll_{i_j} B_j$ for all $j$.
  This satisfies isotony and additivity essentially by construction, and satisfies reflexivity since all the $\ll_i$ do.

  If moreover each $\ll_i$ is topological, then $x\ll_{i_j} \setof{y|y\ll_{i_j} B_j}$ by transitivity for $\ll_{i_j}$, while $\bigcap_{j=1}^n \setof{y|y\ll_{i_j} B_j} \subseteq \setof{y|y\ll A}$ essentially by the above definition of $\ll$; thus $x\ll \setof{y|y\ll A}$.
  %If instead each $\ll_i$ is \nn-stable, then we have $C_j$ with $x\ll_{i_j} \cpl{C_j}\subseteq B_j$; then if $C = \bigcup_{j=1}^n C_j$ we have $\bigcap_{j=1}^n \cpl{C_j} = \cpl{C}$, so $x\ll \cpl{C} \subseteq A$.
  Finally, if each $\ll_i$ is locally decomposable, we have $C_j$ such that $x\ll_{i_j} C_j$ and $\forall y, (y\notin C_j \lor y\ll_{i_j} B_j)$.
  Then $x \ll \bigcap_{j=1}^n C_j$, while any $y$ must either be in all the $B_j$ (hence in $A$) or not in some $C_j$ (hence not in $\bigcap_{j=1}^n C_j$).
  Thus $\bigcap_{j=1}^n C_j$ exhibits local decomposability of $\ll$.

  The argument for apartness is similar: define $x\bowtie A$ to mean that there exist $i_1,\dots,i_n$ and $B_1,\dots, B_n$ such that $A\subseteq \bigcup_{j=1}^n B_j$ and $x\bowtie_{i_j} B_j$ for all $j$.
  Isotony, additivity, and reflexivity are again obvious.
  If each $\bowtie_i$ is topological, then have $x\bowtie_{i_j} \setof{y|y\not\bowtie_{i_j} B_j}$, while $\setof{y|y\not\bowtie A} \subseteq \bigcup_{j=1}^n\setof{y|y\not\bowtie_{i_j} B_j}$ by construction, so $x\bowtie \setof{y|y\not\bowtie A}$.
  If instead all the $\bowtie_i$ are locally decomposable, we then have $C_j$ such that $x\bowtie_{i_j} C_j$ and $\forall y, (y\in C_j \lor y\bowtie_{i_j} B_j)$.
  Then $x\bowtie \bigcup_{j=1}^n C_j$, while any $y$ must be either in some $C_j$ (hence in $\bigcup_{j=1}^n C_j$) or satisfy $y\bowtie_{i_j} B_j$ for all $j$, hence $y\bowtie A$.
  %(If each $\bowtie_i$ is \nn-stable, then $x\bowtie_{i_j} \cpl{\cpl{B_j}}$, but we cannot say that $\cpl{\cpl{A}} \subseteq \bigcup_{j=1}^n \cpl{\cpl{B_j}}$, only $\cpl{\cpl{A}} \subseteq \cpl{\cpl{\bigcup_{j=1}^n B_j}}$.)

  Next, if $(Y,\ll_Y)$ is a pretopological neighborhood space and $f:X\to Y$ is any function, define $x\ll_f A$ to mean that $f\inv(B)\subseteq A$ for some $B$ such that $f(x)\ll_Y B$.
  This is a pretopological neighborhood relation on $X$: isotony is by construction, reflexivity is obvious, and additivity follows since $f\inv$ preserves intersections.
  If $\ll_Y$ is topological and $x\ll_f A$, so that $f(x)\ll_Y B$ and $f\inv(B)\subseteq A$, then $f(x)\ll_Y \setof{y|y\ll_Y B}$; but $f\inv(\setof{y|y\ll_Y B}) \subseteq \setof{z|z\ll_f A}$ essentially by definition of $\ll_f$, so $x \ll_f \setof{z|z\ll_f A}$.
  %If instead $\ll_Y$ is \nn-stable, then $f(x)\ll_Y \cpl{C} \subseteq B$, whence $x \ll_f \cpl{f\inv(C)}$ and $\cpl{f\inv(C)} = f\inv(\cpl{C}) \subseteq f\inv(B) \subseteq A$.
  Finally, if $\ll_Y$ is locally decomposable and, we have a $C$ with $f(x)\ll C$ and $\forall y, (y\notin C \lor y\ll B)$; but then $x\ll_f f\inv(C)$ and $\forall z, (f(z)\notin C \lor f(z) \ll B)$, hence $\forall z, (z\notin f\inv(C) \lor z \ll_f A)$.

  Moreover, $f$ is continuous for some pretopological neighborhood relation $\ll_X$ on $X$ if and only $\ll_f$ is contained in $\ll_X$.
  More generally, for any pretopological neighborhood space $(Z,\ll_Z)$ and function $g:Z\to X$, the composite $f g$ is continuous if and only if $\ll_{f g}$ (which coincides with $(\ll_f)_g$) is contained in $\ll_Z$.
  Thus, any source $\{ f_i : X \to Y_i \}$, where each $Y_i$ is a pretopological neighborhood space, has an initial lift, consisting of the join of all the $\ll_{f_i}$ constructed as above; and if each $Y_i$ is topological %, \nn-stable,
  or locally decomposable, so is the resulting structure on $X$.

  The argument for apartness is analogous, using the relation $x\bowtie_f A$ defined to mean $f(x)\bowtie_Y f(A)$.
  This is a pretopological apartness relation on $X$: reflexivity is obvious, while isotony and additivity follow since direct images of subsets preserve containment and unions.
  If $\bowtie_Y$ is topological and $x\bowtie_f A$, so that $f(x)\bowtie_Y f(A)$, then $f(x) \ll_Y \setof{y|y\not\bowtie_Y f(A)}$; but $f(\setof{z|z\not\bowtie_f A}) \subseteq \setof{y|y\not\bowtie_Y f(A)}$ essentially by definition of $\bowtie_f$, so $x\bowtie_f \setof{z|z\not\bowtie_f A}$.
  If instead $\bowtie_Y$ is locally decomposable, we have a $B$ with $f(x)\bowtie B$ and $\forall y,(y\in B \lor y\bowtie f(A))$.
  Thus $\forall z, (f(z)\in B \lor f(z)\bowtie f(A))$, hence $\forall z, (z\in f\inv(B) \lor z\bowtie_f A)$, while $f(x) \bowtie f(f\inv(B))$ by isotony and thus $x \bowtie_f f\inv(B)$.
  % (Note that this construction does also work for \nn-stability, so that $\APTopnn$ is at least a fibration over $\Set$.)

  Again $f$ is continuous for $\bowtie_X$ iff $\bowtie_f$ is contained in $\bowtie_X$, and similarly for $f g$.
  Thus to construct an initial lift we can join all the $\bowtie_{f_i}$, which is topological or locally decomposable if all the $Y_i$ are.
\end{proof}

Note that an analogous argument for closure spaces \emph{doesn't} work.
This is another reason closure spaces are unsatisfactory.
The close similarity between the proofs for $\bowtie$ and $\ll$ is one reason we have chosen to work with $\ll$ rather than open sets.

For the relationship between neighborhood and apartness spaces, we have the following analogue of \cref{thm:ord-le-apt}:

\begin{thm}\label{thm:top-ll-bowtie}
  If $X$ is a pretopological neighborhood space, define $x\bowtie A$ to mean $x\ll \cpl{A}$; this is a pretopological apartness space $\anti(X)$.
  Dually, if $X$ is a pretopological apartness space, define $x\ll A$ to mean that there exists a $B$ such that $x\bowtie B$ and $\cpl{B}\subseteq A$; this is a pretopological neighborhood space $\neigh(X)$.
  These operations define an idempotent adjunction $\anti\adj \neigh$:
  \[ \anti : \PTop \rightleftarrows \APTop : \neigh. \]
  Moreover:
  \begin{enumerate}
  \item Like any idempotent adjunction, it restricts to an equivalence between the fixed objects on each side.
    A pretopological neighborhood space is fixed if and only if whenever $x\ll A$, there is a subset $B$ such that $x\ll \cpl{B} \subseteq A$.
    A pretopological apartness space is fixed if and only if whenever $x\bowtie A$, we have $x\bowtie \cpl{\cpl{A}}$.
    We call these objects \textbf{\nn-stable}.
  \item If $X$ is a topological neighborhood space, then $\anti(X)$ is a topological apartness space.
  \item If $X$ is a topological apartness space, then it is \nn-stable and hence a fixed point of the adjunction.
  \item Both functors $\anti$ and $\neigh$ preserve local decomposability, and locally decomposable spaces are \nn-stable.
    Thus the adjunction restricts to an equivalence $\ldTop \simeq \ldATop$.
    \label{item:top-ll-bowtie-equiv}
  \end{enumerate}
\end{thm}
\begin{proof}
  In both cases the preservation of isotony, reflexivity, and additivity is easy.
  There are no worries about de Morgan's law, because in \emph{both} cases the set we are complementing is the one apart from something, binary additivity for apartness involves unions, and $\cpl{(A\cup B)} = \cpl{A} \cap \cpl{B}$ is the de Morgan law that does hold constructively.

  For functoriality of $\anti$, suppose $f:X\to Y$ is a continuous map of pretopological neighborhood spaces, and that $f(x) \bowtie f(A)$, meaning by definition $f(x)\ll \cpl{f(A)}$.
  Then by continuity of $f$ we have $x \ll f\inv(\cpl{f(A)}) = \cpl{f\inv(f(A))}$, hence $x\bowtie f\inv(f(A))$ by definition.
  But $A \subseteq f\inv(f(A))$, so $x\bowtie A$ by isotony.
  Thus $\anti(f)$ is continuous.

  For functoriality of $\neigh$, suppose $f:X\to Y$ is a continuous map of pretopological apartness spaces, and that $f(x) \ll A$, meaning that $f(x) \bowtie B$ and $\cpl{B} \subseteq A$ for some $B$.
  Then $f(x) \bowtie f(f\inv(B))$ by isotony, so $x\bowtie f\inv(B)$ by continuity of $f$, while $\cpl{f\inv(B)} = f\inv(\cpl{B}) \subseteq f\inv(A)$; so $x\ll f\inv(A)$ by definition.
  Thus $\neigh(f)$ is continuous.

  As in \cref{thm:ord-le-apt}, the unit and counit are the identity functions, so that the triangle identities are trivial once we know that the unit and counit are continuous.
  If $X$ is a pretopological neighborhood space, then in $\neigh(\anti(X))$ to say ``$x\ll A$'' means that there is a $B$ such that $\cpl{B}\subseteq A$ and $x\ll \cpl{B}$.
  This implies $x\ll A$ in $X$ by isotony, so the identity function $X\to \neigh(\anti(X))$ is continuous; and it is an isomorphism exactly when $X$ is \nn-stable as defined above.
  
  On the other side, if $X$ is a pretopological apartness space, then in $\anti(\neigh(X))$ to say ``$x\bowtie A$'' means that $x\bowtie B$ and $\cpl{B}\subseteq \cpl{A}$ for some $B$.
  This is implied by $x\bowtie A$ in $X$ (take $B=A$), so the identity function $\anti(\neigh(X)) \to X$ is continuous.
  To see that it is an isomorphism exactly when $X$ is \nn-stable as defined above, note that $\cpl{B}\subseteq \cpl{A}$ is equivalent to $A \subseteq \cpl{\cpl{B}}$.

  For idempotence, it suffices to show that $\anti(X)$ and $\neigh(X)$ are both \nn-stable.
  In the former case this follows from the identity $\cpl{\cpl{\cpl{A}}} = \cpl{A}$, and in the latter essentially by definition.

  To see that $\anti$ preserves transitivity, suppose $x\bowtie A$ in $\anti(X)$, i.e.\ $x\ll \cpl{A}$.
  To show $x\bowtie \setof{y|\neg(y\bowtie A)}$ means to show $x\ll \setof{y|\neg\neg(y\bowtie A)} = \setof{y|\neg\neg(y\ll \cpl{A})}$.
  But $x\ll \setof{y|y\ll \cpl{A}}$ by transitivity for $\ll$, and $(y\ll \cpl{A})\implies\neg\neg(y\ll\cpl{A})$, so the result follows by isotony.

  A locally decomposable neighborhood space is \nn-stable since if $\forall y, (y\notin B \lor y \ll A)$ then $\neg\neg B \subseteq A$.
  Similarly, a topological apartness space is \nn-stable, since $\cpl{\cpl{A}}\subseteq \setof{ y | y \not\bowtie A }$ by the contrapositive of reflexivity.
  Thus, it remains to show that $\anti$ and $\neigh$ both preserve local decomposability.

  Suppose first that $X$ is a locally decomposable neighborhood space, and $x \bowtie A$ in $\anti(X)$, so that $x\ll \cpl{A}$.
  Then we have a $B$ with $x\ll B$ and $\forall y, (y\notin B \lor y\ll \cpl{A})$.
  Let $C = \cpl{B}$; then $x\ll \cpl{C}$ by isotony, i.e.\ $x\bowtie C$.
  Moreover, for any $y$ we have either $y\ll\cpl{A}$, hence $y\bowtie A$, or $y\notin B$, hence $y\in C$.
  Thus, $\anti(X)$ is locally decomposable.

  In the other direction, suppose $X$ is a locally decomposable apartness space, and $x\ll A$ in $\neigh(X)$, so that $x\bowtie B$ and $\cpl{B}\subseteq A$.
  Then we have a $C$ with $x\bowtie C$ and $\forall y, (y\in C \lor y\bowtie B)$.
  Let $D = \cpl{C}$; then $x\ll D$ since $x\bowtie C$.
  Moreover, for any $y$ we have either $y\in C$, hence $y\notin D$, or $y\bowtie B$, hence $y\ll A$.
  Thus, $\neigh(X)$ is locally decomposable.
\end{proof}

\begin{rmk}
  In particular, it follows that \nn-stable pretopological neighborhood spaces are a reflective subcategory of $\PTop$, while \nn-stable pretopological apartness spaces are a coreflective subcategory of $\APTop$.
  In particular, both are complete and cocomplete, and indeed topological over $\Set$.
  % \cref{thm:top-ll-bowtie} explains why \cref{thm:top-topconcrete} does not include $\APTopnn$: rather than being reflective in $\APTop$, the category $\APTopnn$ is \emph{coreflective} therein.
  % This implies that it is also topological over $\Set$ in its own right, though it is not closed under initial lifts in $\APTop$.
\end{rmk}

Recall that in \cref{thm:pmet-top} we showed that a prometric space has both a locally decomposable neighborhood relation and a locally decomposable apartness.

\begin{thm}\label{thm:promet-top}
  The $\bowtie$ and $\ll$ underlying any prometric space are identified with each other by the equivalence of \cref{thm:top-ll-bowtie}\ref{item:top-ll-bowtie-equiv}.
  Moreover, this defines a fully faithful functor from prometric spaces and continuous functions to $\ldTop$ (or equivalently $\ldATop$).
\end{thm}
\begin{proof}
  Let $X$ be a prometric space; we first show that $x\bowtie A$ if and only if $x\ll \cpl{A}$ according to \cref{defn:toprels}, so that the two structures are identified by $\anti$.
  If $x\bowtie A$, we have $d,\ep$ such that $y\in A \implies d(x,y)>\ep$ for any $y$.
  By a contrapositive, it follows that $d(x,y)\le \ep \implies y\notin A$, giving $x\ll \cpl{A}$.
  Conversely, if $x\ll\cpl{A}$, we have $d,\ep$ such that $d(x,y)< \ep \implies y\notin A$ for any $y$.
  The contrapositive is now $\neg\neg(y\in A) \implies d(x,y)\ge \ep$, which implies $y\in A \implies d(x,y)> \hfep$, so $x\bowtie A$.

  Now let $X$ and $Y$ be prometric spaces and $f:X\to Y$ a function.
  Suppose first that $f$ is prometrically continuous, and let $f(x)\ll A$, so that we have $d_Y,\ep$ with $d_Y(f(x),y)< \ep \implies y\in A$ for any $y\in Y$.
  By continuity, there is a $d_X$ and a $\de$ such that $d_X(x,x')<\de \implies d_Y(f(x),f(x'))<\ep$ for any $x'\in X$, and hence $d_X(x,x')<\de \implies x'\in f\inv(A)$.
  Thus $x\ll f\inv(A)$, so $f$ is $\ll$-continuous.

  On the other hand, if $f$ is $\ll$-continuous, for any $x$ and any $d_Y,\ep$ we have $f(x)\ll \setof{y|d_Y(f(x),y)<\ep}$.
  Thus $x\ll \setof{x'|d_Y(f(x),f(x'))<\ep}$, which means (by definition of $\ll$) we have $d_X,\de$ such that $d_X(x,x')<\de \implies d_Y(f(x),f(x'))<\ep$, giving prometric continuity.
\end{proof}

Thus, both locally decomposable neighborhood spaces and locally decomposable apartness spaces suffice to capture the topological information in a prometric space.
Subsequently we may simply speak of a \textbf{locally decomposable topological space}, which comes with both a $\ll$ and a $\bowtie$ that are interdefinable.

\begin{rmk}
  \cref{thm:top-topconcrete} implies that the forgetful functor from locally decomposable spaces to sets has a left adjoint, assigning to every set the finest possible locally decomposable topology on that set.
  This is \emph{not} generally the discrete topology, since a discrete topology (meaning $x\ll A$ iff $x\in A$) is locally decomposable if and only if equality on $X$ is decidable.
  Of course we cannot expect to prove that this topology is non-discrete in any particular case, since excluded middle implies that it is.
  But it seems likely that in many interesting cases it will turn out metatheoretically to coincide with the ``natural'' topology.
\end{rmk}


\subsection{Separation axioms}
\label{sec:separation}

However, prometric spaces (especially symmetric ones and gauge spaces) are still very special objects of $\ldTop$, and much of their specialness is captured by additional \emph{separation axioms}.
(Local decomposability itself is also naturally regarded as a separation axiom, albeit one that is only nontrivial constructively.
This will be clearer in later sections, when its close relationship with regularity becomes evident.)

We will not exhaustively survey separation axioms, but there are a few others worth mentioning.
The first few are naturally phrased using the fact that point-set relations give rise to point-point relations.

%TODO: Left adjoints
% giving a left adjoint to the functor taking a topological neighborhood space to its underlying $\leapx$.
%This likewise gives a left adjoint to the functor taking a pretopological apartness relation to its underlying $\oapt$ (which is only irreflexive in the pretopological case), and 

\begin{thm}\label{thm:top-ord}\ 
  \begin{enumerate}
  \item If $X$ is a topological neighborhood space, define $x\leapx y$ to mean that $\forall A, (x\ll A \to y\ll A)$.
    Then $\leapx$ is a preorder, called the \textbf{specialization order} of $X$.
  \item If $X$ is a topological apartness space, define $x\oapt y$ to mean $x \bowtie \singleton{y}$.
    Then $\oapt$ is an anti-preorder, called the \textbf{specialization anti-preorder} of $X$.
    If $X$ is locally decomposable, then $\oapt$ is a comparison.
  \end{enumerate}
  Moreover, if $X$ is a prometric space then both of these relations agree with those defined in \cref{def:promet-ord}.
\end{thm}
\begin{proof}
  In the neighborhood case, reflexivity and transitivity of $\leapx$ follow directly from reflexivity and transitivity of $\implies$.
  If $X$ is a prometric space and $\forall d, d(x,y)=0$, then whenever $x\ll A$ we have $d,\ep$ with $d(x,z)<\ep \implies z\in A$ for any $z$.
  Let $d'$ be as in prometric transitivity for $d$; then $d'(x,y)=0$, so if $d'(y,z)<\ep$ we have $d(x,z)<\ep$, hence $z\in A$.
  Thus $d',\ep$ exhibit $y\ll A$.

  Conversely, if $\forall A, (x\ll A \implies y\ll A)$, then for any $d,\ep$ let $A = \setof{z|d(x,z)<\ep}$.
  Then $x\ll A$, so $y\ll A$, and in particular $y\in A$, so $d(x,y)<\ep$.
  Since this is true for all $\ep$, we have $d(x,y)=0$ for all $d$.

  In the apartness case, irreflexivity is obvious.
  For bilocal transitivity, if $x\oapt z$ so that $x\bowtie \singleton{z}$, then by transitivity for $\bowtie$ we have $x \bowtie \setof{y|y\not \bowtie \singleton{z}}$.
  Thus, if $\neg (y\oapt z)$, so that $\neg (y\bowtie \singleton{z})$, then $y$ belongs to a set that is apart from $x$; hence $x\bowtie \singleton{y}$ by isotony, i.e.\ $x\oapt y$.

  For the comparison axiom, we note first that local decomposability implies the following property that we call \textbf{weak local decomposability}: if $x\bowtie A$, then $\forall y, (x\bowtie \singleton{y} \lor y\bowtie A)$.
  For if $x\bowtie B$ and $y\in B$ as in local decomposability, then $x\bowtie \singleton{y}$.
  Moreover, weak local decomposability implies that $\oapt$ is a comparison, by taking $A = \singleton{y}$.

  Finally, if $X$ is a prometric space and we have a $d$ with $d(x,y)>0$, then taking $\ep=\frac12 d(x,y)$ we find $x\bowtie \singleton{y}$.
  Conversely, if $x\bowtie \singleton{y}$, then we have $d,\ep$ with $\forall z, (z\in \singleton{y} \implies d(x,z)>\ep)$, which is to say $d(x,y)>\ep$, hence $d(x,y)>0$.
\end{proof}

\begin{defn}
  Let $X$ be a topological neighborhood space.
  \begin{enumerate}
  \item $X$ is $\mathbf{T_0}$ if $(x\leapx y \land y\leapx x) \implies x=y$ (i.e.\ $\leapx$ is antisymmetric).
  \item $X$ is $\mathbf{T_1}$ if $x\leapx y \implies x=y$ (i.e.\ $\leapx$ coincides with equality).
  \item $X$ is $\mathbf{R_0}$ if $\leapx$ is symmetric (hence an equivalence relation).
  \end{enumerate}
  Let $X$ be a topological apartness space.
  \begin{enumerate}
  \item $X$ is $\mathbf{T_0}$ if $(\neg(x\oapt y) \land \neg(y\oapt x)) \implies x=y$ (i.e.\ $\oapt$ is connected).
  \item $X$ is $\mathbf{T_1}$ if $\neg(x\oapt y)\implies x=y$ (i.e.\ $\oapt$ is tight).
  % \item $X$ is $\apart$-$\mathbf{T_1}$ if $x\apart y \implies x\oapt y$, for some tight relation $\apart$.
  \item $X$ is \textbf{strongly} $\mathbf{R_0}$ if $\oapt$ is symmetric (hence, if $X$ is also locally decomposable, $\oapt$ is an apartness relation).
  \end{enumerate}
\end{defn}

\begin{thm}
  Let $X$ be a prometric space.
  \begin{enumerate}
  \item $X$ is $T_1$ or $T_0$ if and only if its underlying topology is so.
  \item If $X$ is symmetric, then its underlying topology is strongly $R_0$.
  \end{enumerate}
\end{thm}
\begin{proof}
  Obvious.
\end{proof}

\begin{thm}\label{thm:ldtop-ord}
  If $X$ is a locally decomposable topological space, then for any $x,y\in X$ we have
  \[ (x\leapx y) \iff \neg(x\oapt y) \]
  and the two definitions of $T_1$ and $T_0$ above agree, while strong $R_0$ implies $R_0$.
\end{thm}
\begin{proof}
  First let $x\leapx y$, so that $\forall A, (x\ll A \implies y\ll A)$, and suppose for contradiction that $x\oapt y$, i.e.\ $x\bowtie \singleton{y}$.
  The latter means by definition that $x\ll \cpl{\singleton{y}}$, so by assumption we have $y\ll \cpl{\singleton{y}}$, a contradiction since $y\notin \cpl{\singleton{y}}$.

  On the other hand, suppose $\neg(x\oapt y)$, and let $x\ll A$; we will show $y\ll A$.
  By local decomposability we have a $B$ with $x\ll B$ and $\forall z, (z\ll A \lor z\notin B)$.
  Thus, to show $y\ll A$ it suffices to show that $y\notin B$ leads to a contradiction.
  But if $y\notin B$ then $B\subseteq \cpl{\singleton{y}}$, so $x\ll \cpl{\singleton{y}}$ by isotony and thus $x\bowtie \singleton{y}$, contradicting $\neg(x\oapt y)$.

  The remaining statements are obvious.
\end{proof}

In keeping with our categorical perspective, we observe also that \cref{thm:ldtop-ord} can be promoted to a natural transformation.

\begin{thm}
  Let $\mathbf{Refl}$ and $\mathbf{Irref}$ denote the categories of reflexive and irreflexive relations respectively, with morphisms defined to preserve and anti-preserve the relation as in $\Preord$ and $\APreord$ respectively.
  Then the constructions of \cref{thm:top-ord} generalize to functors $\PTop \to \mathbf{Refl}$ and $\APTop \to \mathbf{Irref}$, and we have a mate-pair of natural transformations lying over the identity in $\Set$:
  \[
  \xymatrix{ \PTop \ar[r] \ar[d]_{\anti} \drtwocell\omit & \mathbf{Refl} \ar[d]^{\anti} \\
    \APTop \ar[r]  & \mathbf{Irref}  }
  \qquad
  \xymatrix{ \PTop \ar[r]  & \mathbf{Refl}   \\
    \APTop \ar[r] \ar[u]^{\neigh} \urtwocell\omit & \mathbf{Irref}\ar[u]_{\neigh} }
  \]
  of which the second becomes an isomorphism when restricted to locally decomposable spaces:
  \[ \xymatrix{ \ldTop \ar[r]  & \Preord   \\
    \ldATop \ar[r] \ar[u]^{\neigh}_{\simeq} \urtwocell\omit{\cong} & \APreord \ar[u]_{\neigh} }
  \]
\end{thm}
\begin{proof}
  First let $f:X\to Y$ be a continuous map of pretopological neighborhood spaces, and $x\leapx y$ in $X$.
  If $f(x)\ll A$ in $Y$, then $x\ll f\inv(A)$ in $X$, hence $y\ll f\inv(A)$ since $x\leapx y$, and thus $f(y)\ll A$.
  Thus $f(x)\leapx f(y)$, and we have a functor $\PTop \to \mathbf{Refl}$.

  Second, let $f:X\to Y$ be a continuous map of pretopological apartness spaces, and $f(x)\oapt f(y)$ in $Y$.
  Then $f(x) \bowtie \singleton{ f(y)} = f(\singleton{y})$, so $x\bowtie \singleton{y}$ and $x\oapt y$.
  Thus we have a functor $\APTop \to \mathbf{Irref}$.

  By the mate-correspondence, it suffices to construct the right-hand transformation.
  Thus let $X$ be a pretopological apartness space.
  In the putative domain of the transformation, $x\leapx y$ means $\forall A, (x\ll A \implies y\ll A)$ in $\neigh(X)$, while in the putative codomain, $x\leapx y$ means $\neg (x\bowtie \singleton{y})$.
  Assume the former, and for contradiction that $x\bowtie \singleton{y}$.
  Then $x\ll \cpl{\singleton{y}}$ in $\neigh(X)$, whence $y\ll \cpl{\singleton{y}}$, but this is a contradiction since $y\notin \cpl{\singleton{y}}$.

  The fact that this becomes an isomorphism on locally decomposable spaces is the $\Leftarrow$ direction of \cref{thm:ldtop-ord}.
\end{proof}

In fact, the functors $\PTop \to \mathbf{Refl}$ and $\APTop \to \mathbf{Irref}$ defined above have left adjoints:

\begin{eg}\label{eg:alexandrov}
  Any reflexive relation $\leapx$ induces a pretopological neighborhood relation where $x\ll A$ means $\forall y, (x\leapx y \implies y\in A)$.
  When $\leapx$ is transitive and hence a preorder, this $\ll$ is topological; indeed it is the standard \textbf{Alexandrov topology} of a preorder.
  The functor taking a reflexive relation or preorder to its Alexandrov (pre)topology is left adjoint to the functor $\PTop \to \mathbf{Refl}$ or $\Top \to \Preord$, and the spaces obtained in this way are precisely those with the property that whenever $x\ll A_i$ for all $i$ we have $x\ll \bigcap_i A_i$.
  Note that the Alexandrov topology is locally decomposable if and only if $\leapx$ is decidable, and that it includes the case of discrete topologies when $\leapx$ is equality.
  
  % isotony and additivity are obvious, while reflexivity for $\ll$ follows from reflexivity for $\leapx$.
  % For transitivity, if $x\ll A$, so that $\forall y, (x\leapx y \implies y\in A)$, to show $x\ll \setof{y|y\ll A}$ we must show that if $x\leapx y$ then $y\ll A$, i.e.\ if $y\leapx z$ then $z\in A$.
  % But transitivity of $\leapx$ gives $x\leapx z$, so $z\in A$ by assumption.
  % Note that this \emph{Alexandrov neighborhood space} is locally decomposable if and only if $\leapx$ is decidable, i.e.\ $\forall x\forall y, (x\leapx y \lor x\not\leapx y )$.

  % If $X$ is a preorder and $Y$ a topological neighborhood space, then $f:X\to Y$ is continuous for this structure if and only if $f(x) \ll A$ implies $\forall y, (x\leapx y \implies y\in f\inv(A))$.
  % Put differently, this means that if $x\leapx y$, then $f(x)\ll A$ implies $f(y)\ll A$, i.e.\ $x\leapx y \implies f(x)\leapx_Y f(y)$, where $\leapx_Y$ is defined as in \cref{thm:top-ord}.
  % This gives adjointness.

  Dually, any irreflexive relation $\oapt$ induces a pretopological apartness relation, where $x\bowtie A$ means $\forall y, (y\in A \implies x\oapt y)$, which we might call the \textbf{Alexandrov apartness}.
  This defines a left adjoint to the functor $\APTop \to \mathbf{Irref}$, and the apartness spaces obtained in this way are precisely those in which $x\bowtie A_i$ for all $i$ implies $x\bowtie \bigcup_i A_i$.
  However, the Alexandrov apartness is not usually topological; even the comparison property $x\oapt y \lor y\oapt z$ for $\oapt$ seems insufficient, constructively, to ensure transitivity of $\bowtie$.
\end{eg}

The other pair of separation axioms we will be interested in are regularity and complete regularity.
We define them only for neighborhood spaces, since as we will see below they both imply local decomposability.

\begin{defn}
  Let $X$ be a topological neighborhood space.
  \begin{enumerate}
  \item $X$ is \textbf{regular} if whenever $x\ll A$, there are a $B$ and $C$ such that $x\ll B$, $B\cap C = \emptyset$, and $\forall y, (y\ll A \lor y\ll C)$.
  \item $X$ is \textbf{completely regular} if whenever $x\ll A$, there is a continuous function $f:X\to \R$ such that $f(x)=0$ and $\forall y, (f(y)<1 \implies y\in A)$.
  \end{enumerate}
\end{defn}

\begin{lem}\label{thm:creg-reg}
  A completely regular space is regular.
\end{lem}
\begin{proof}
  Given $x\ll A$, find $f$ as in complete regularity, and let $B = \setof{y|f(y)<\frac12}$ and $C = \setof{y|f(y)>\frac12}$.
  Then $x\ll B$ by continuity of $f$ and the assumption $f(x)=0$, and clearly $B\cap C = \emptyset$.
  For any $y$, we have either $f(y)>\frac12$ or $f(y)<1$.
  In the former case, we have $y \ll C$ by continuity.
  In the latter case, we have $y \ll \setof{z|f(z)<1}$ by continuity, hence $y\ll A$ by isotony and the assumption on $f$.
\end{proof}

\begin{thm}
  A regular space is both locally decomposable and strongly $R_0$.
\end{thm}
\begin{proof}
  Local decomposability is clear since $B\cap C=\emptyset$ means that $y\ll C \implies y\notin B$.
  For strong $R_0$, suppose $x\oapt y$, i.e.\ $x\bowtie \singleton{y}$, i.e.\ $x\ll \cpl{\singleton{y}}$.
  By regularity we have $B,C$ with $x\ll B$, $B\cap C = \emptyset$, and $\forall z, (z\ll \cpl{\singleton{y}} \lor z\ll C)$.
  Since $y\not\ll \cpl{\singleton{y}}$, we have $y\ll C$; but $x\notin C$ so $C\subseteq \cpl{\singleton{x}}$.
  Thus $y\ll\cpl{\singleton{x}}$ by isotony, so $y\oapt x$.
\end{proof}

\begin{thm}\label{thm:sympmet-reg}
  If $X$ is a symmetric prometric space, then its underlying topology is regular.
\end{thm}
\begin{proof}
  Given $x\ll A$, we have $d_1,\ep$ such that $\forall y, (d_1(x,y)<\ep \implies y\in A)$.
  Let $d_2$ be as in prometric transitivity for $d_1$, and define $B = \setof{y|d_2(x,y)<\hfep}$ and $C = \setof{y|d_2(x,y)>\hfep}$.
  Then clearly $x\ll B$ and $B\cap C = \emptyset$.

  Now for any $y$, we have either $d_2(x,y)<\frac{2\ep}3$ or $d_2(x,y)>\hfep$.
  First suppose $d_2(x,y)<\frac{2\ep}3$; we claim $y\ll A$.
  Suppose $z$ is such that $d_2(y,z)<\frac{\ep}{3}$; then $d_1(x,z) \le d_2(x,y)+d_2(y,z) < \frac{2\ep}{3} + \frac{\ep}{3} = \ep$.
  Hence, by assumption, $z\in A$.
  Thus $\forall z, (d_2(y,z)<\frac{\ep}{3} \implies z\in A)$, so $y\ll A$.

  Now suppose $d_2(x,y)>\hfep$; we claim $y\ll C$.
  Let $0<\ep' < d_2(x,y)-\hfep$, let $d_3$ be as in prometric transitivity for $d_2$, and let $d_4$ be as in prometric symmetry for $d_3$.
  Suppose $z$ is such that $d_4(y,z)<\ep'$; then $d_3(z,y)<\ep'$.
  Moreover, $d_2(x,y) \le d_3(x,z) + d_3(z,y) < d_3(x,z) + \ep'$, so $d_3(x,z) > d_2(x,y) - \ep' > \hfep$.
  Thus $z\in C$.
  Therefore $\forall z, (d_4(y,z)<\ep' \implies z\in C)$, so $y\ll C$.
\end{proof}

Note the similarity between the proofs of \cref{thm:creg-reg} and \cref{thm:sympmet-reg}: in the latter we use a distance whereas in the former we use an unknown continuous function.
This suggests that prometric spaces should in fact be completely regular with distances playing the role of the separating functions; but only in the gauge case can we show that distances are continuous.

\begin{thm}\label{thm:symgauge-creg}
  If $X$ is a (symmetric) gauge space, then its underlying topology is completely regular.
\end{thm}
\begin{proof}
  Given $x\ll A$, we have a gauging distance $d$ and an $\eta>0$ such that $\forall y, (d(x,y)<\eta \implies y\in A)$.
  Let $f(y) = \frac1{\eta} d(x,y)$; it clearly suffices to show that $f$ is continuous.
  In fact we will show that $f$ is uniformly continuous.
  Thus, suppose given $\ep$; we must find a gauging distance $d'$ and a $\de>0$ such that for any $y,z\in X$, if $d'(y,z)<\de$ then $|f(y)-f(z)|<\ep$.
  Let $d'$ be a gauging distance such that $d(u,v)\le d'(u,v)$ and also $d(v,u) \le d'(u,v)$ for all $u,v\in X$, which exists by symmetry and binary filtration, and let $\de = \eta\ep$.
  Suppose $d'(y,z)< \de= \eta\ep$; then by the triangle inequality we have
  \begin{gather*}
    d(x,z) \le d(x,y)+d(y,z) < d(x,y) + d'(y,z) < d(x,y)+\eta\ep\\
    d(x,y) \le d(x,z)+d(z,y) < d(x,z) + d'(y,z) < d(x,z)+\eta\ep
  \end{gather*}
  Thus
  \[ |f(y)-f(z)| = |\textstyle\frac1{\eta}d(x,y) - \frac1{\eta}d(x,z)| < \ep \]
  as desired.
\end{proof}

In classical mathematics, prometric spaces are also completely regular (with a less direct proof), since they are uniformizable (see \cref{sec:uniformity}).
But in the absence of excluded middle and countable choice, this seems no longer true.

Classically, complete regularity also completely determines the gauge spaces: every completely regular space can be given by a gauge.
\fxnote{Is that also true constructively?}

\cref{thm:sympmet-reg} and \cref{thm:symgauge-creg} do depend crucially on symmetry.
Classically, \emph{any} topological space can be given by a quasi-gauge (although constructively, even a prometric space satisfies the nontrivial condition of local decomposability).


\subsection{Comparing definitions}
\label{sec:top-compare}

Our definition of topological apartness space is clearly inspired by the point-set apartness spaces of Bridges and Vita~\cite[Chapter 2]{bridges-vita}, but there are several minor differences.

Firstly, they assume $X$ to be inhabited, whereas we follow instead the lesson of category theory that trivial examples should not be arbitrarily excluded.
In particular, an assumption of inhabitedness would destroy \cref{thm:top-topconcrete} by removing the initial object.

Secondly, they assume that the set $X$ comes with a given symmetric irreflexive relation $\apart$ (they write it as $\neq$), and modify various of the axioms to refer to this relation rather than the denial inequality $\neg(x=y)$ (which, recally, is what \emph{we} mean by ``$x\neq y$'').
In particular, their reflexivity axiom is
\begin{equation}
  x\bowtie A \implies \forall y\in A, (x\apart y)\label{eq:top-apart-refl}
\end{equation}
which implies ours since $\apart$ is irreflexive, and reduces to ours if $\apart$ is $\neq$.
Note that this is equivalent to
\[ (x\oapt y) \implies (x\apart y) \]
where $\oapt$ is defined as in \cref{thm:top-ord}.
Thus, if in any pretopological apartness space (in our sense) we define $x\apart y$ to mean $(x\oapt y)\lor (y\oapt x)$, we get a symmetric irreflexive relation satisfying~\eqref{eq:top-apart-refl}.
We tend to think that not much generality is lost by defining $\apart$ in terms of the topology this way, rather than postulating it from the beginning; this is further supported by the fact that their $T_1$ separation axiom is
\[ (x\apart y) \implies (x\oapt y) \]
so that in a $T_1$ apartness space as they define it, their postulated $\apart$ coincides with our defined $\oapt$.
Their $T_1$ axiom implies ours as long as $\apart$ is itself tight.

Thirdly, their definition of a point-set apartness space (as opposed to what they call a ``pre-apartness space'') also includes the following axiom:
\[ x\bowtie A \implies \forall y, (x\apart y \lor y\bowtie A). \]
If $\apart$ coincides with $\oapt$, then this axiom becomes the statementthat in the proof of \cref{thm:top-ord} we called \emph{weak local decomposability}.
As with local decomposability, we regard it as a separation axiom rather than something that should be part of a definition.

Finally, and most importantly, they use a weaker transitivity axiom:
\begin{equation}
  x\bowtie A \implies x \bowtie \setof{ y | \forall z, (z\bowtie A \implies z\apart y)}.\label{eq:top-wktrans}
\end{equation}
This is clearly implied by our transitivity axiom, since $\forall z, (z\bowtie A \implies z\apart y)$ implies $y\not\bowtie A$.
However, it seems too weak even in classical mathematics.
For instance, if $X = \{x,y,z\}$ with an irreflexive relation defined by $x\apart z$ and $z\apart x$ and nothing else, then the corresponding Alexandrov pretopological apartness space (\cref{eg:alexandrov}) satisfies~\eqref{eq:top-wktrans}; but classically it is a pretopological space that is not topological.
Moreover,~\eqref{eq:top-wktrans} also destroys \cref{thm:top-topconcrete}; indeed it is not even inherited by subspaces.

It is worth noting that in the presence of~\eqref{eq:top-wktrans} and~\eqref{eq:top-apart-refl}, our stronger transitivity axiom is equivalent to what~\cite{bridges-vita} calls the \textbf{reverse Kolmogorov property}: whenever $x\bowtie A$ and $y\not\bowtie A$ then $x\apart y$.
This is equivalent to saying $\setof{y|y\not\bowtie A} \subseteq \setof{ y | \forall z, (z\bowtie A \implies z\apart y)}$, which together with~\eqref{eq:top-wktrans} and isotony implies our transitivity axiom.
On the other hand, assuming our transitivity axiom, if $x\bowtie A$ and $y\not\bowtie A$, then since $x\bowtie \setof{y|y\not\bowtie A}$, hence by~\eqref{eq:top-apart-refl} we have $x\apart y$.
Thus, the differences are fairly minor.

If $\apart$ coincides with $\neq$, then our functor $\anti$ constructs what~\cite{bridges-vita} call the \emph{topological apartness} associated to a topological space.
Our functor $\neigh$ is not mentioned therein.
Its adjointness to $\anti$ makes it clearly of categorical interest, but it is also obviously problematic in that it doesn't preserve transitivity.
Instead they use the following construction, called the \emph{apartness topology}:

\begin{thm}\label{thm:aptop}
  In any pretopological apartness space $X$, define $x\ll A$ to mean
  \begin{equation}
    \label{eq:top-bowtie-ll-1}
    \exists B, (x\bowtie B \land \forall y, (y\bowtie B \implies y\in A)).
  \end{equation}
  Then $X$ becomes a topological neighborhood space, denoted $\neigh'(X)$.
  Moreover:
  \begin{enumerate}
  \item If $X$ is a topological apartness space, then $\anti(\neigh'(X)) \cong X$.\label{item:aptop1}
  \item If $X$ is a \nn-stable topological neighborhood space, then $\neigh'(\anti(X)) \cong X$.\label{item:aptop2}
  \end{enumerate}
\end{thm}
\begin{proof}
  Isotony is clear, and additivity follows from additivity for $\bowtie$.
  For transitivity, note that $x\ll A$ if and only if there is a set of the form $\setof{y|y\bowtie B}$ containing $x$ and contained in $A$.
  Thus, $\setof{y|y\ll A}$ is the union of all sets of the form $\setof{y|y\bowtie B}$ contained in $A$.
  Since any such set is also contained in that union, if $x\ll A$ we have $x\ll \setof{y|y\ll A}$, as desired.
  (In more traditional language, the sets $\setof{y|y\bowtie B}$ are a basis for the topology of $\neigh'(X)$.)

  Now by definition we have $x\bowtie A$ in $\anti(\neigh'(X))$ if and only if $x\ll \cpl{A}$ in $\neigh'(X)$, meaning there is a $B$ such that $x\bowtie B$ in $X$ and $\forall y, (y\bowtie B \implies y\in \cpl{A})$.
  This certainly holds if $x\bowtie A$: take $B=A$.
  But conversely, it implies $A\subseteq \setof{y|y\not\bowtie B}$, and $x\bowtie\setof{y|y\not\bowtie B}$ by transitivity for $X$, so $x\bowtie A$ by isotony; this gives~\ref{item:aptop1}.

  Finally, for~\ref{item:aptop2}, if $x \ll A$ in $\neigh'(\anti(X))$, then there is a $B$ such that $x\ll \cpl{B}$ and $\forall y, (y\ll \cpl{B} \implies y\in A)$, i.e.\ $x\in \int(\cpl{B}) \subseteq A$.
  This certainly implies $x\ll A$ (as long as $X$ is topological), while the converse follows from \nn-stability.
\end{proof}

Unfortunately, $\neigh'$ does not seem to be a functor $\ATop\to \Top$ at all.
If $f:X\to Y$ is $\bowtie$-continuous and $f(x)\ll A$ in $\neigh' Y$, then we have $B\subseteq Y$ with $f(x)\bowtie B$ and $\forall y\in Y, (y\bowtie B \implies y\in A)$.
Since $f(f\inv(B)) \subseteq B$, we have $f(x)\bowtie f(f\inv(B))$ and thus $x\bowtie f\inv(B)$, while for all $z\in X$ we have $f(z)\bowtie B \implies f(z)\in A$, i.e.\ $f(z)\bowtie B \implies z\in f\inv(A)$.
But $f(z)\bowtie B$ is not implied by $z\bowtie f\inv(B)$.
If we replace $B$ by $\setof{y|y\not\bowtie B}$, allowing us to assume that $\forall y, (y\not\bowtie B \implies y\in B)$, we can conclude that $f(z)\not\bowtie B$ implies $f(z)\in B$, so that $z\bowtie f\inv(B)$ implies $f(z) \notin B$ and hence $\neg(f(z)\not\bowtie B)$, but this is still not yet $f(z)\bowtie B$.

However, it is easy to check that if $X$ is a locally decomposable apartness space, then $\neigh(X) \cong \neigh'(X)$, and both definitions of $x\ll A$ are also equivalent to:
\begin{equation}
    %\exists B, (x\bowtie B \land \forall y, (y\bowtie B \implies y\in A))\\%\label{eq:top-bowtie-ll-1}\\
  \exists B, (x\bowtie B \land \forall y, (y\in A \lor y\in B)).\label{eq:top-bowtie-ll-2}
\end{equation}

\begin{rmk}
  The frequent use of the non-functorial construction $\neigh'$ in~\cite[Chapter 2]{bridges-vita} does tend to make a category theorist suspicious.
  For instance, they define:
  \begin{itemize}
  \item \emph{convergence} in an apartness space $X$ to be simply topological convergence in $\neigh'(X)$,
  \item the \emph{weak nested neighborhoods} property for an apartness space $X$ to be essentially \nn-stability of $\neigh'(X)$, and
  \item the \emph{product} of apartness spaces $X$ and $Y$ to be essentially $\anti(\neigh'(X) \times \neigh'(Y))$.
  \end{itemize}
  To a category theorist, this suggests that the ``real'' objects of interest in~\cite[Chapter 2]{bridges-vita} are the topological \emph{neighborhood} spaces of the form $\neigh'(X)$ for some apartness space $X$, or equivalently those such that $\neigh'(\anti(X))\cong X$.
  In~\cite{bridges-vita} such spaces are called \emph{topologically consistent}: they are the topological spaces in which the interiors of complements form a basis for the topology.
\end{rmk}


\section{Families of point-point relations: uniformity}
\label{sec:uniformity}

\cref{thm:promet-top} shows that topological neighborhood or apartness spaces are sufficient to capture the topological information in a prometric space.
However, there is also the more refined \emph{uniform} information, as defined by the category of uniformly continuous functions.
The standard way to divorce uniformity from real numbers is to consider \emph{families} of point-point relations.

\begin{defn}\label{def:pmet-unif}
  Let $X$ be a prometric space, $d$ a distance, $\ep>0$ a real number, and $x,y\in X$.
  \begin{enumerate}
  \item We say $x \ent{d,\ep} y$ if $d(x,y)<\ep$.
  \item We say $x \aent{d,\ep} y$ if $d(x,y)>\ep$.
  \end{enumerate}
\end{defn}

The abstract structures capturing these relations are the following.
We will use Greek letters $\al,\be,\gm$ to denote special binary relations containing the diagonal, with $(x,y)\in \al$ written as $x\ent\al y$.
Similarly, we will use $\rho,\si,\tau$ to denote special binary relations disjoint from the diagonal, with $
(x,y)\in\rho$ written $x \aent\rho y$.

We write $\neg\al$ for the complement of a relation $\al$, for which we also invert the notation $\leapx/\oapt$.
That is, $(x \aent{\neg\al} y)$ means $\neg (x\ent{\al} y)$ and similarly.
We write $\al\implies\be$ to mean that $\forall x\forall y,(x\ent\al y \implies x\ent\be y)$ or similarly (i.e.\ $\al\subseteq \be$ as subsets of $X\times X$).

\begin{defn}
  A \textbf{quasi-uniform space} is a set $X$ together with a family of binary relations, called \textbf{entourages}, satisfying the following axioms.
  \begin{enumerate}
  \item If $\al$ is an entourage and $\al\implies\be$, then $\be$ is an entourage (isotony).
  \item There exists an entourage (nullary filtration).
  \item If $\al$ and $\be$ are entourages, there exists an entourage $\gm$ such that $\gm\implies\al$ and $\gm\implies\be$ (binary filtration).
  \item We have $x\ent\al x$ for all entourages $\al$ and all $x\in X$ (reflexivity).
  \item For any entourage $\al$, there exists an entourage $\be$ such that for any $x,y,z\in X$, if $x\ent\be y$ and $y\ent\be z$, then $x\ent\al z$ (transitivity).
  \end{enumerate}
  A quasi-uniform space is\dots
  \begin{enumerate}[resume]
  \item \textbf{uniform} if for any entourage $\al$, there is an entourage $\be$ such that for any $x,y\in X$, if $x\ent\be y$ then $y\ent\al x$ (symmetry).
    By isotony, there is then a base (see below) of symmetric entourages, which we usually write $\approx_\al$ instead of $\ent\al$.
  % \end{enumerate}
  % A quasi-uniform space is\dots
  % \begin{enumerate}[resume]
  % \item \textbf{\nn-stable} if for any entourage $\al$, there is an entourage $\be$ such that $\neg\neg\be\implies \al$. %, i.e.\ $\forall x\forall y, (\neg\neg(x\ent\be y) \implies x\ent\al y)$.
  \item \textbf{locally decomposable} if for any entourage $\al$, there is an entourage $\be$ such that for any $x,y\in X$, either $x\ent\al y$ or $\neg(x\ent\be y)$.
  \end{enumerate}
  If $X$ and $Y$ are quasi-uniform spaces, a function $f:X\to Y$ is \textbf{uniformly continuous} if for any entourage $\al$ in $Y$ there is an entourage $\be$ in $X$ such that $\forall x\forall y, (x\ent\be y \implies f(x)\ent\al f(y))$.
  We write $\QUnif$ for the category of quasi-uniform spaces and uniformly continuous maps, and similarly
  % $\Unifnn$ and
  $\Unif$ and $\ldUnif$ for its full subcategories of uniform %, \nn-stable,
  and locally decomposable quasi-uniform spaces.
\end{defn}

Uniform spaces are well-known in classical mathematics.
Constructively, some authors such as~\cite{bridges-vita} include uniform local decomposability as part of the definition, but as with topological local decomposability, we regard it as a separation axiom to be imposed only when required.
For instance, any topological group is a uniform space in the usual way, but it need not be locally decomposable.
Likewise, any set has a discrete uniformity in which all reflexive relations are entourages, but this is locally decomposable only when the set has decidable equality.

Less well-known is the dual notion of uniform apartness.

\begin{defn}
  A \textbf{quasi-uniform apartness space} (or \textbf{anti-quasi-uniform space}) is a set $X$ together with a family of binary relations, called \textbf{anti-entourages}, satisfying the following axioms.
  \begin{enumerate}
  \item If $\rho$ is an anti-entourage and $\si\implies\rho$, then $\si$ is an anti-entourage (isotony).
  \item There exists an anti-entourage (nullary filtration).
  \item If $\rho$ and $\si$ are anti-entourages, there exists an anti-entourage $\tau$ such that $\rho\implies\tau$ and $\si\implies\tau$ (binary filtration).
  \item We have $\neg(x\aent\rho x)$ for all anti-entourages $\rho$ and all $x\in X$ (reflexivity).
  \item For any anti-entourage $\rho$, there exists an anti-entourage $\si$ such that for any $x,y,z\in X$, if $x\aent\rho z$ and $\neg(y\aent\si z)$, then $x\aent\si y$ (bilocal transitivity).
  \end{enumerate}
  A quasi-uniform apartness space is\dots
  \begin{enumerate}[resume]
  \item a \textbf{uniform apartness space} (or \textbf{anti-uniform space}) if for any anti-entourage $\rho$, there is an anti-entourage $\si$ such that for any $x,y\in X$, if $x\aent\rho y$ then $y\aent\si x$ (symmetry).
    By isotony, there is then a base (see below) of symmetric anti-entourages, which we usually write $\apart_\rho$.
  % \end{enumerate}
  % A quasi-uniform apartness space is\dots
  % \begin{enumerate}[resume]
  % \item \textbf{\nn-stable} if for any anti-entourage $\rho$, there is an entourage $\si$ such that $\neg\neg\rho \implies \si$. %, i.e.\ $\forall x\forall y, (\neg\neg(x\aent\rho y) \implies x\aent\si y)$.
  \item \textbf{locally decomposable} if for any anti-entourage $\rho$, there is an anti-entourage $\si$ such that for any $x,y\in X$, either $x\aent\si y$ or $\neg(x\aent\rho y)$.
    Note that then automatically $\rho\implies\si$.
  \item \textbf{bilocally decomposable} if for any anti-entourage $\rho$, there is an anti-entourage $\si$ such that for any $x,y,z\in X$, if $x\aent\rho z$ then either $x\aent\si y$ or $y\aent\si z$.
  \end{enumerate}
  If $X$ and $Y$ are quasi-uniform apartness spaces, a function $f:X\to Y$ is \textbf{uniformly continuous} if for any anti-entourage $\al$ in $Y$ there is an anti-entourage $\be$ in $X$ such that $\forall x\forall y, (f(x)\aent\al f(y) \implies x\aent\be y)$.
  We write $\AQUnif$ for the category of quasi-uniform apartness spaces and uniformly continuous maps, and similarly $\AUnif$ % $\AUnifnn$
  and $\ldAQUnif$ for its full subcategories of uniform %,  \nn-stable
  and locally decomposable quasi-uniform apartness spaces.
\end{defn}

Note that the ``bilocal transitivity'' condition for a uniform apartness is weaker than what we have called bilocal decomposability.
Just as a quasi-uniform space with a smallest entourage can be identified with a preorder, an anti-quasi-uniform space with a largest anti-entourage can be identified with an anti-preorder (\cref{def:anti-preorder}).
However, we note:

\begin{lem}
  If a quasi-uniform apartness space is locally decomposable, then it is bilocally decomposable.
\end{lem}
\begin{proof}
  Given $\rho$, let $\si$ be as in transitivity, and then let $\tau$ be as in local decomposability for $\si$.
  Now suppose $x\aent\rho z$.
  We have either $\neg(y\aent\si z)$ or $y\aent\tau z$; but in the former case transitivity gives $x\aent\si y$.
  Since $\si\implies\tau$, we have $x\aent\tau y$ or $y\aent\tau z$, as desired.
\end{proof}

As usual, a \textbf{base} for a uniformity is a subset of ``basic'' entourages such that every entourage contains a basic one.
Dually, a base for an anti-uniformity is a subset of basic anti-entourages such that every anti-entourage is contained in a basic one.
A set of entourages or anti-entourages is a base for \emph{some} uniformity or anti-uniformity exactly when it satisfies all of the axioms except isotony.
(We have stated the axioms in such a way as to make this true, though in the presence of isotony the filtration axioms have a simpler form.)

\begin{thm}\label{thm:pmet-unif}
  For any prometric space $X$, \cref{def:pmet-unif} gives bases for a locally decomposable quasi-uniformity and a locally decomposable anti-quasi-uniformity, which are symmetric if $X$ is.
\end{thm}
\begin{proof}
  In both cases nullary filtration follows from nullary filtration of distances.
  For binary filtration, given $(d_1,\ep_1)$ and $(d_2,\ep_2)$, let $d_3$ be as in binary filtration for distances; then $(d_3,\min(\ep_1,\ep_2))$ is what we want.
  Both reflexivity axioms follow directly from reflexivity for distances, and likewise symmetry follows from symmetry for distances (when it holds).
  For uniform transitivity, given $(d,\ep)$ let $d'$ be as in prometric transitivity and consider $(d',\hfep)$: if $d'(x,y)<\hfep$ and $d'(y,z)<\hfep$, then $d(x,z) \le d'(x,y)+d'(y,z)<\ep$.

  For anti-uniform transitivity, given $(d,\ep)$ let $d'$ be as in prometric transitivity and consider $(d',\fep13)$.
  If $d(x,z)>\ep$ and $\neg (d'(y,z)>\fep13)$, then $d'(y,z)\le \fep13$.
  Now $d'(x,y)$ is either $>\fep13$ or $<\fep23$; but in the latter case we have $d(x,z) \le d'(x,y)+d'(y,z) < \ep$, a contradiction.

  Finally, for uniform local decomposability, given $(d,\ep)$ consider $(d,\hfep)$;
  then for any $x,y$ we have $d(x,y)<\ep$ or $d(x,y)>\hfep$.
  The anti-uniform case is basically identical.
\end{proof}

\begin{thm}
  The categories $\QUnif$ and $\AQUnif$ are both topological over $\Set$.
  Moreover, their subcategories $\Unif$, $\ldQUnif$, $\AUnif$, and $\ldAQUnif$ are closed under initial lifts of sources, hence are also topological over $\Set$ and are reflective in their supercategory.
\end{thm}
\begin{proof}
  TODO
\end{proof}

\begin{thm}
  If $X$ is a quasi-uniform space, the complements $\neg\al$ of entourages form a base for an anti-quasi-uniformity.
  Dually, if $X$ is a quasi-uniform apartness space, the complements $\neg\rho$ of the anti-entourages form a base for a quasi-uniformity.
  These operations define an idempotent adjunction
  \[ \anti : \QUnif \rightleftarrows \AQUnif : \neigh. \]
  Moreover:
  \begin{enumerate}
  \item The fixed quasi-uniform spaces are those such that for any entourage $\al$, there is an entourage $\be$ such that $\neg\neg\be\implies \al$.
    The fixed quasi-uniform apartness spaces are those such that for any anti-entourage $\rho$, there is an entourage $\si$ such that $\neg\neg\rho \implies \si$.
    We call these objects \textbf{\nn-stable}.
  \item Both $\anti$ and $\neigh$ preserve local decomposability.
  \item Locally decomposable (anti)-quasi-uniform spaces are \nn-stable, so we have an induced equivalence $\ldQUnif \simeq \ldAQUnif$.
  \item Both $\anti$ and $\neigh$ preserve symmetry, so the above adjunction and equivalences restrict from the quasi-uniform case to the uniform case.
  \end{enumerate}
\end{thm}
\begin{proof}
  TODO
\end{proof}

\begin{thm}
  For any prometric space $X$, the locally decomposable quasi-uniform and anti-quasi-uniform structures of \cref{thm:pmet-unif} correspond under the equivalence $\ldQUnif \simeq \ldAQUnif$.
  Moreover, this defines a fully faithful functor from prometric spaces and uniformly continuous functions to $\ldQUnif$ (or equivalently $\ldAQUnif$).
\end{thm}
\begin{proof}
  It suffices to show that for any distance $d$ and $\ep>0$, there exists a distance $d'$ and an $\ep'>0$ such that
  \[\neg(x\aent{d',\ep'} y) \implies x\ent{d,\ep} y \quad\text{and}\quad x\aent{d,\ep} y \implies \neg(x\ent{d',\ep'} y) .\]
  Take $d'=d$ and $\ep'= \hfep$; the first statement follows from $d(x,y)<\ep \lor d(x,y)>\hfep$, while the second is obvious.

  Now if $X$ and $Y$ are prometric spaces and $f:X\to Y$ is a function, to say that it is uniformly continuous for the induced quasi-uniformities means that for any basic entourage defined by $(d_Y,\ep)$ on $Y$, there exists a basic entourage defined by $(d_X,\de)$ on $X$ such that $x\ent{d_X,\de} y \implies f(x) \ent{d_Y,\ep} f(y)$.
  But this is exactly uniform continuity as defined for prometric spaces in \cref{sec:metric}.
\end{proof}

Thus, locally decomposable (quasi-)uniform spaces capture exactly the uniform information in a prometric space.
Classically, \emph{every} quasi-uniform space can be given a prometric, and indeed a quasi-gauge, and similarly in the symmetric case; but this seems not to be true constructively.
(It is true, however, if we allow distances to take values in \emph{upper} real numbers, since then from any entourage $\al$ we can define a distance by $d(x,y) = \inf\setof{ 0 | x\ent\al y}$.
Similarly, if we allow distances to take values in \emph{lower} real numbers, any bilocally decomposable\fxnote{Is bilocal decomposability necessary and sufficient for this?} anti-quasi-uniform space becomes quasi-gaugeable, since from any anti-entourage $\rho$ we can define a distance by $d(x,y) = \sup\setof{ q\in\mathbb{Q} | x\aent\rho y }$.)

\begin{thm}
  TODO: Underlying topology of uniform structures, how it commutes with the adjunctions $\anti\dashv \neigh$ and preserves local decomposability.
\end{thm}

\begin{thm}
  TODO: The orders underlying the uniform topology are the intersection of entourages and union of anti-entourages, giving a co/reflection from uniformities into orders.
\end{thm}

Is it worth introducing, for pedagogical reasons, the notion of a set equipped with a jointly-transitive filter of pretopologies (i.e.\ a left-perfect syntopogeny) as a common generalization of topologies and uniformities, before we introduce proximities and go all the way to general syntopogenies?
They do include \emph{approach spaces} and I guess \emph{pro-approach spaces} as further generalization of both.

\begin{thm}
  The topology underlying a locally decomposable (symmetric) uniform space is regular.
\end{thm}

TODO: More about uniform regularity, as discussed on the nLab and nForum.


\section{Set-set relations: proximity}
\label{sec:set-set}
\label{sec:proximity}

Although somewhat less well-known, there is an intermediate notion between topology and uniformity, generally known as \emph{proximity}~\cite{proximity-spaces}.
Like topologies, proximities can be defined classically by three equivalent kinds of relation --- neighborhood, apartness, and closeness --- but now these are relations between pairs of sets rather than between a point and a set.
In the prometric case, the relevant relations are:

\begin{defn}
  Let $X$ be a prometric space and $A,B\subseteq X$.
  \begin{enumerate}
  \item We say $A\ll B$ if $\exists d, \exists \ep, \forall x, \forall y, (x\in A \land d(x,y)<\ep \implies y\in B)$.
  \item We say $A\bowtie B$ if $\exists d, \exists \ep, \forall x, \forall y, (x\in A \land y\in B \implies d(x,y)>\ep)$.
  \item We say $A\approx B$ if $\forall d, \forall \ep, \exists x, \exists y, (x\in A \land y\in B \land d(x,y)<\ep)$.
  \end{enumerate}
\end{defn}

We now proceed to axiomatize these relations.

\begin{defn}
  A \textbf{pre-quasi-proximal neighborhood space} is a set $X$ with a binary relation $\ll$ on subsets such that
  \begin{enumerate}
  \item If $A \subseteq B \ll C \subseteq D$, then $A\ll D$ (isotony).
  \item If $A\ll B$ then $A\subseteq B$ (reflexivity).
  \item $\emptyset \ll A$ and $A\ll X$ for all $A$ (nullary additivity).
  \item If $A\ll B$ and $A\ll C$, then $A\ll B\cap C$ (right binary additivity).
  \item If $A\ll C$ and $B\ll C$, then $A\cup B \ll C$ (left binary additivity).
  \end{enumerate}
  A pre-quasi-proximal neighborhood space is\dots
  \begin{enumerate}[resume]
  \item \textbf{quasi-proximal} if whenever $A\ll B$, there is a $C$ such that $A\ll C \ll B$ (transitivity).
  \item \textbf{pre-proximal} if whenever $A\ll B$ we have $\cpl{B} \ll \cpl{A}$ (symmetry).
  \item \textbf{proximal} if it satisfies both transitivity and symmetry.
  \item \textbf{decomposable} if whenever $A\ll B$, there exist $C$ and $D$ such that $A\ll C$, $D\ll B$, and $\cpl{C}\cup D = X$.
  \end{enumerate}
  \textbf{proximally continuous}
\end{defn}

\begin{defn}
  A \textbf{pre-quasi-proximal apartness space} is a set $X$ with a binary relation $\bowtie$ on subsets such that
  \begin{enumerate}
  \item If $A\subseteq B \bowtie C \supset D$, then $A\bowtie D$ (isotony).
  \item If $A\bowtie B$ then $A\cap B = \emptyset$ (reflexivity).
  \item $\emptyset \bowtie A$ and $A\bowtie \emptyset$ for all $A$ (nullary additivity).
  \item 
  \item 
  \end{enumerate}
  A pre-quasi-proximal apartness space is\dots
  \begin{enumerate}[resume]
  \item 
  \item 
  \item 
  \item 
  \end{enumerate}
  \textbf{proximally continuous}
\end{defn}

\begin{defn}
  \textbf{quasi-proximal closeness space}
\end{defn}


\section{Families of set-set relations: syntopogeny}
\label{sec:syntopogeny}



% \section{Located subspaces}
% \label{sec:located}

% \begin{defn}
%   A subset $A\subseteq X$ is \textbf{(metrically) located} if for all $x\in X$, the set $\setof{ d(x,y) | y\in A }$ has an greatest lower bound, written $d(x,A)$.
%   \fxwarning{Which direction does this go in the quasi case?}
% \end{defn}

% Classically, the poset $\Rpd$ is a complete lattice, and thus every subset is located.
% (Even the empty set: $d(x,\emptyset)=\infty$.
% This is one good reason for allowing infinite distances.)
% However, constructively there are ``wild'' subsets such as $\setof{x\in X | P}$, for some undecidable proposition $P$, that cannot be shown to be located in general.
% \fxnote{Mention some theorems about located subsets.}

\section{Completeness}
\label{sec:completeness}

Among the notions that are invariants of continuity or uniform continuity are convergence and completeness.
When there is only one prometric space $X$, we assume that the letter $d$ ranges over distances on $X$.

\begin{itemize}
\item $X$ is \textbf{totally bounded} if for all $d$ and $\ep>0$ there is a finite set $F\subseteq X$ such that for all $y\in X$ there is an $x\in F$ such that $d(x,y)<\ep$.
\item A sequence $\{x_n\}$ in $X$ \textbf{converges} to $x\in X$ if for every $d$ and $\ep>0$ there is an $N$ such that for any $n$, if $n>N$ then $d(x,x_n)<\ep$.
\item More generally, a filter $\F$ in $X$ \textbf{converges} to $x\in X$ if for every $d$ and $\ep>0$ there is an $A\in\F$ such that for all $y\in A$ we have $d(x,y)<\ep$.
\item A sequence $\{x_n\}$ in $X$ is \textbf{Cauchy} if for every $d$ and $\ep>0$ there is an $N$ such that for any $n,m$, if $n,m>N$ then $d(x_n,x_m)<\ep$.
\item A filter $\F$ in $X$ is \textbf{Cauchy} if for every $d$ and $\ep>0$ there is an $A\in\F$ such that for all $x,y\in A$ we have $d(x,y)<\ep$.
\item $X$ is \textbf{complete} if every Cauchy filter converges to some point.
\end{itemize}

For metric spaces in classical mathematics, it suffices to consider convergent and Cauchy \emph{sequences}.
However, in the gauge and prometric cases we must deal with filters (or nets); and constructively (specifically, in the absence of both excluded middle and countable choice) this is true even for metric spaces.
For instance, the completion of the metric space $\mathbb{Q}$ under limits of Cauchy sequences (the ``Cauchy real numbers'') is potentially much smaller than its full completion under limits of Cauchy filters (the Dedekind real numbers $\R$).
In fact, there may not even be ``enough'' Cauchy filters, so that the completion is properly a non-spatial locale; we will return to this in \cref{sec:completion}.

Note also that the above definitions of Cauchy filter and complete space are only really sensible in the symmetric case.
We will also return to this later.\fxnote{Do that}

In classical mathematics, a metric space is \emph{compact} if and only if it is complete and totally bounded.
Constructively, this is not so; thus one generally either uses ``complete and totally bounded'' as a replacement for ``compact'', or passes to locales instead of spaces (see \cref{sec:locales}).

% Classically, one can \emph{complete} a metric space by equipping a set of equivalence classes of Cauchy sequences with a metric.



\bibliographystyle{alpha}
\bibliography{syntop}

\end{document}
