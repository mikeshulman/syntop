\documentclass{article}
\usepackage{amssymb,amsmath,stmaryrd,mathrsfs}

%% Set this to true before loading if we're using the TAC style file.
%% Note that eventually, TAC requires everything to be in one source file.
\def\definetac{\newif\iftac}    % Can't define a \newif inside another \if!
\ifx\tactrue\undefined
  \definetac
  %% Guess whether we're using TAC by whether \state is defined.
  \ifx\state\undefined\tacfalse\else\tactrue\fi
\fi

% Similarly detect beamer
\def\definebeamer{\newif\ifbeamer}
\ifx\beamertrue\undefined
  \definebeamer
  %% Guess whether we're using Beamer by whether \uncover is defined.
  \ifx\uncover\undefined\beamerfalse\else\beamertrue\fi
\fi

% And cleveref
\def\definecref{\newif\ifcref}
\ifx\creftrue\undefined
  \definecref
  % Default to false
  \creffalse
\fi

\iftac\else\usepackage{amsthm}\fi
\usepackage[all,2cell]{xy}
%\UseAllTwocells
%\usepackage{tikz}
%\usetikzlibrary{arrows}
\ifbeamer\else
  \usepackage{enumitem}
  \usepackage{xcolor}
  \definecolor{darkgreen}{rgb}{0,0.45,0} 
  \iftac\else\usepackage[pagebackref,colorlinks,citecolor=darkgreen,linkcolor=darkgreen]{hyperref}\fi
\fi
\usepackage{mathtools}          % for all sorts of things
\usepackage{graphics}           % for \scalebox, used in \widecheck
\usepackage{ifmtarg}            % used in \jd
\usepackage{microtype}
%\usepackage{color,epsfig}
%\usepackage{fullpage}
%\usepackage{eucal}
%\usepackage{wasysym}
%\usepackage{txfonts}            % for \invamp, or for the nice fonts
\usepackage{braket}             % for \Set, etc.
\let\setof\Set
\usepackage{url}                % for citations to web sites
\usepackage{xspace}             % put spaces after a \command in text
%\usepackage{cite}               % compress and sort grouped citations (only use with numbered citations)
\ifcref\usepackage{cleveref,aliascnt}\fi

%% If you want to use biblatex, e.g. if a journal requires (Author name YEAR) citations.
% \usepackage[style=authoryear,
%  backref=true,
%  maxnames=4,
%  maxbibnames=99,
%  uniquename=false,
%  firstinits=true
% ]{biblatex}
% \addbibresource{all.bib}

% \let\cite\parencite
% \DeclareNameAlias{sortname}{last-first}

\makeatletter
\let\ea\expandafter

%% Defining commands that are always in math mode.
\def\mdef#1#2{\ea\ea\ea\gdef\ea\ea\noexpand#1\ea{\ea\ensuremath\ea{#2}\xspace}}
\def\alwaysmath#1{\ea\ea\ea\global\ea\ea\ea\let\ea\ea\csname your@#1\endcsname\csname #1\endcsname
  \ea\def\csname #1\endcsname{\ensuremath{\csname your@#1\endcsname}\xspace}}

%% WIDECHECK
\DeclareRobustCommand\widecheck[1]{{\mathpalette\@widecheck{#1}}}
\def\@widecheck#1#2{%
    \setbox\z@\hbox{\m@th$#1#2$}%
    \setbox\tw@\hbox{\m@th$#1%
       \widehat{%
          \vrule\@width\z@\@height\ht\z@
          \vrule\@height\z@\@width\wd\z@}$}%
    \dp\tw@-\ht\z@
    \@tempdima\ht\z@ \advance\@tempdima2\ht\tw@ \divide\@tempdima\thr@@
    \setbox\tw@\hbox{%
       \raise\@tempdima\hbox{\scalebox{1}[-1]{\lower\@tempdima\box
\tw@}}}%
    {\ooalign{\box\tw@ \cr \box\z@}}}

%% SIMPLE COMMANDS FOR FONTS AND DECORATIONS

\newcount\foreachcount

\def\foreachletter#1#2#3{\foreachcount=#1
  \ea\loop\ea\ea\ea#3\@alph\foreachcount
  \advance\foreachcount by 1
  \ifnum\foreachcount<#2\repeat}

\def\foreachLetter#1#2#3{\foreachcount=#1
  \ea\loop\ea\ea\ea#3\@Alph\foreachcount
  \advance\foreachcount by 1
  \ifnum\foreachcount<#2\repeat}

% Script: \sA is \mathscr{A}
\def\definescr#1{\ea\gdef\csname s#1\endcsname{\ensuremath{\mathscr{#1}}\xspace}}
\foreachLetter{1}{27}{\definescr}
% Calligraphic: \cA is \mathcal{A}
\def\definecal#1{\ea\gdef\csname c#1\endcsname{\ensuremath{\mathcal{#1}}\xspace}}
\foreachLetter{1}{27}{\definecal}
% Bold: \bA is \mathbf{A}
\def\definebold#1{\ea\gdef\csname b#1\endcsname{\ensuremath{\mathbf{#1}}\xspace}}
\foreachLetter{1}{27}{\definebold}
% Blackboard Bold: \dA is \mathbb{A}
\def\definebb#1{\ea\gdef\csname d#1\endcsname{\ensuremath{\mathbb{#1}}\xspace}}
\foreachLetter{1}{27}{\definebb}
% Fraktur: \fa is \mathfrak{a}, except for \fi; \fA is \mathfrak{A}
\def\definefrak#1{\ea\gdef\csname f#1\endcsname{\ensuremath{\mathfrak{#1}}\xspace}}
\foreachletter{1}{9}{\definefrak}
\foreachletter{10}{27}{\definefrak}
\foreachLetter{1}{27}{\definefrak}
% Sans serif: \ia is \mathsf{a}, except for \if and \in
\def\definesf#1{\ea\gdef\csname i#1\endcsname{\ensuremath{\mathsf{#1}}\xspace}}
\foreachletter{1}{6}{\definesf}
\foreachletter{7}{14}{\definesf}
\foreachletter{15}{27}{\definesf}
\foreachLetter{1}{27}{\definesf}
% Bar: \Abar is \overline{A}, \abar is \overline{a}
\def\definebar#1{\ea\gdef\csname #1bar\endcsname{\ensuremath{\overline{#1}}\xspace}}
\foreachLetter{1}{27}{\definebar}
\foreachletter{1}{8}{\definebar} % \hbar is something else!
\foreachletter{9}{15}{\definebar} % \obar is something else!
\foreachletter{16}{27}{\definebar}
% Tilde: \Atil is \widetilde{A}, \atil is \widetilde{a}
\def\definetil#1{\ea\gdef\csname #1til\endcsname{\ensuremath{\widetilde{#1}}\xspace}}
\foreachLetter{1}{27}{\definetil}
\foreachletter{1}{27}{\definetil}
% Hats: \Ahat is \widehat{A}, \ahat is \widehat{a}
\def\definehat#1{\ea\gdef\csname #1hat\endcsname{\ensuremath{\widehat{#1}}\xspace}}
\foreachLetter{1}{27}{\definehat}
\foreachletter{1}{27}{\definehat}
% Checks: \Achk is \widecheck{A}, \achk is \widecheck{a}
\def\definechk#1{\ea\gdef\csname #1chk\endcsname{\ensuremath{\widecheck{#1}}\xspace}}
\foreachLetter{1}{27}{\definechk}
\foreachletter{1}{27}{\definechk}
% Underline: \uA is \underline{A}, \ua is \underline{a}
\def\defineul#1{\ea\gdef\csname u#1\endcsname{\ensuremath{\underline{#1}}\xspace}}
\foreachLetter{1}{27}{\defineul}
\foreachletter{1}{27}{\defineul}

% Particular commands for typefaces, sometimes with the first letter
% different.
\def\autofmt@n#1\autofmt@end{\mathrm{#1}}
\def\autofmt@b#1\autofmt@end{\mathbf{#1}}
\def\autofmt@d#1#2\autofmt@end{\mathbb{#1}\mathsf{#2}}
\def\autofmt@c#1#2\autofmt@end{\mathcal{#1}\mathit{#2}}
\def\autofmt@s#1#2\autofmt@end{\mathscr{#1}\mathit{#2}}
\def\autofmt@f#1\autofmt@end{\mathsf{#1}}
\def\autofmt@k#1\autofmt@end{\mathfrak{#1}}
% Particular commands for decorations.
\def\autofmt@u#1\autofmt@end{\underline{\smash{\mathsf{#1}}}}
\def\autofmt@U#1\autofmt@end{\underline{\underline{\smash{\mathsf{#1}}}}}
\def\autofmt@h#1\autofmt@end{\widehat{#1}}
\def\autofmt@r#1\autofmt@end{\overline{#1}}
\def\autofmt@t#1\autofmt@end{\widetilde{#1}}
\def\autofmt@k#1\autofmt@end{\check{#1}}

% Defining multi-letter commands.  Use this like so:
% \autodefs{\bSet\cCat\cCAT\kBicat\lProf}
\def\auto@drop#1{}
\def\autodef#1{\ea\ea\ea\@autodef\ea\ea\ea#1\ea\auto@drop\string#1\autodef@end}
\def\@autodef#1#2#3\autodef@end{%
  \ea\def\ea#1\ea{\ea\ensuremath\ea{\csname autofmt@#2\endcsname#3\autofmt@end}\xspace}}
\def\autodefs@end{blarg!}
\def\autodefs#1{\@autodefs#1\autodefs@end}
\def\@autodefs#1{\ifx#1\autodefs@end%
  \def\autodefs@next{}%
  \else%
  \def\autodefs@next{\autodef#1\@autodefs}%
  \fi\autodefs@next}

%% FONTS AND DECORATION FOR GREEK LETTERS

%% the package `mathbbol' gives us blackboard bold greek and numbers,
%% but it does it by redefining \mathbb to use a different font, so that
%% all the other \mathbb letters look different too.  Here we import the
%% font with bb greek and numbers, but assign it a different name,
%% \mathbbb, so as not to replace the usual one.
\DeclareSymbolFont{bbold}{U}{bbold}{m}{n}
\DeclareSymbolFontAlphabet{\mathbbb}{bbold}
\newcommand{\dDelta}{\ensuremath{\mathbbb{\Delta}}\xspace}
\newcommand{\done}{\ensuremath{\mathbbb{1}}\xspace}
\newcommand{\dtwo}{\ensuremath{\mathbbb{2}}\xspace}
\newcommand{\dthree}{\ensuremath{\mathbbb{3}}\xspace}

% greek with bars
\newcommand{\albar}{\ensuremath{\overline{\alpha}}\xspace}
\newcommand{\bebar}{\ensuremath{\overline{\beta}}\xspace}
\newcommand{\gmbar}{\ensuremath{\overline{\gamma}}\xspace}
\newcommand{\debar}{\ensuremath{\overline{\delta}}\xspace}
\newcommand{\phibar}{\ensuremath{\overline{\varphi}}\xspace}
\newcommand{\psibar}{\ensuremath{\overline{\psi}}\xspace}
\newcommand{\xibar}{\ensuremath{\overline{\xi}}\xspace}
\newcommand{\ombar}{\ensuremath{\overline{\omega}}\xspace}

% greek with tildes
\newcommand{\altil}{\ensuremath{\widetilde{\alpha}}\xspace}
\newcommand{\betil}{\ensuremath{\widetilde{\beta}}\xspace}
\newcommand{\gmtil}{\ensuremath{\widetilde{\gamma}}\xspace}
\newcommand{\phitil}{\ensuremath{\widetilde{\varphi}}\xspace}
\newcommand{\psitil}{\ensuremath{\widetilde{\psi}}\xspace}
\newcommand{\xitil}{\ensuremath{\widetilde{\xi}}\xspace}
\newcommand{\omtil}{\ensuremath{\widetilde{\omega}}\xspace}

% MISCELLANEOUS SYMBOLS
\let\del\partial
\mdef\delbar{\overline{\partial}}
\let\sm\wedge
\newcommand{\dd}[1]{\ensuremath{\frac{\partial}{\partial {#1}}}}
\newcommand{\inv}{^{-1}}
\newcommand{\dual}{^{\vee}}
\mdef\hf{\textstyle\frac12 }
\mdef\thrd{\textstyle\frac13 }
\mdef\qtr{\textstyle\frac14 }
\let\meet\wedge
\let\join\vee
\let\dn\downarrow
\newcommand{\op}{^{\mathrm{op}}}
\newcommand{\co}{^{\mathrm{co}}}
\newcommand{\coop}{^{\mathrm{coop}}}
\let\adj\dashv
\SelectTips{cm}{}
\newdir{ >}{{}*!/-10pt/\dir{>}}    % extra spacing for tail arrows in XYpic
\makeatother
\newcommand{\pushout}[1][dr]{\save*!/#1+1.2pc/#1:(1,-1)@^{|-}\restore}
\newcommand{\pullback}[1][dr]{\save*!/#1-1.2pc/#1:(-1,1)@^{|-}\restore}
\makeatletter
\let\iso\cong
\let\eqv\simeq
\let\cng\equiv
\mdef\Id{\mathrm{Id}}
\mdef\id{\mathrm{id}}
\alwaysmath{ell}
\alwaysmath{infty}
\let\oo\infty
\alwaysmath{odot}
\def\frc#1/#2.{\frac{#1}{#2}}   % \frc x^2+1 / x^2-1 .
\mdef\ten{\mathrel{\otimes}}
\let\bigten\bigotimes
\mdef\sqten{\mathrel{\boxtimes}}
\def\lt{<}                      % For iTex compatibility
\def\gt{>}

%% OPERATORS
\DeclareMathOperator\lan{Lan}
\DeclareMathOperator\ran{Ran}
\DeclareMathOperator\colim{colim}
\DeclareMathOperator\coeq{coeq}
\DeclareMathOperator\eq{eq}
\DeclareMathOperator\Tot{Tot}
\DeclareMathOperator\cosk{cosk}
\DeclareMathOperator\sk{sk}
%\DeclareMathOperator\im{im}
\DeclareMathOperator\Spec{Spec}
\DeclareMathOperator\Ho{Ho}
\DeclareMathOperator\Aut{Aut}
\DeclareMathOperator\End{End}
\DeclareMathOperator\Hom{Hom}
\DeclareMathOperator\Map{Map}

%% ARROWS
% \to already exists
\newcommand{\too}[1][]{\ensuremath{\overset{#1}{\longrightarrow}}}
\newcommand{\ot}{\ensuremath{\leftarrow}}
\newcommand{\oot}[1][]{\ensuremath{\overset{#1}{\longleftarrow}}}
\let\toot\rightleftarrows
\let\otto\leftrightarrows
\let\Impl\Rightarrow
\let\imp\Rightarrow
\let\toto\rightrightarrows
\let\into\hookrightarrow
\let\xinto\xhookrightarrow
\mdef\we{\overset{\sim}{\longrightarrow}}
\mdef\leftwe{\overset{\sim}{\longleftarrow}}
\let\mono\rightarrowtail
\let\leftmono\leftarrowtail
\let\cof\rightarrowtail
\let\leftcof\leftarrowtail
\let\epi\twoheadrightarrow
\let\leftepi\twoheadleftarrow
\let\fib\twoheadrightarrow
\let\leftfib\twoheadleftarrow
\let\cohto\rightsquigarrow
\let\maps\colon
\newcommand{\spam}{\,:\!}       % \maps for left arrows
\def\acof{\mathrel{\mathrlap{\hspace{3pt}\raisebox{4pt}{$\scriptscriptstyle\sim$}}\mathord{\rightarrowtail}}}

% diagxy redefines \to, along with \toleft, \two, \epi, and \mon.

%% EXTENSIBLE ARROWS
\let\xto\xrightarrow
\let\xot\xleftarrow
% See Voss' Mathmode.tex for instructions on how to create new
% extensible arrows.
\def\rightarrowtailfill@{\arrowfill@{\Yright\joinrel\relbar}\relbar\rightarrow}
\newcommand\xrightarrowtail[2][]{\ext@arrow 0055{\rightarrowtailfill@}{#1}{#2}}
\let\xmono\xrightarrowtail
\let\xcof\xrightarrowtail
\def\twoheadrightarrowfill@{\arrowfill@{\relbar\joinrel\relbar}\relbar\twoheadrightarrow}
\newcommand\xtwoheadrightarrow[2][]{\ext@arrow 0055{\twoheadrightarrowfill@}{#1}{#2}}
\let\xepi\xtwoheadrightarrow
\let\xfib\xtwoheadrightarrow
% Let's leave the left-going ones until I need them.

%% EXTENSIBLE SLASHED ARROWS
% Making extensible slashed arrows, by modifying the underlying AMS code.
% Arguments are:
% 1 = arrowhead on the left (\relbar or \Relbar if none)
% 2 = fill character (usually \relbar or \Relbar)
% 3 = slash character (such as \mapstochar or \Mapstochar)
% 4 = arrowhead on the left (\relbar or \Relbar if none)
% 5 = display mode (\displaystyle etc)
\def\slashedarrowfill@#1#2#3#4#5{%
  $\m@th\thickmuskip0mu\medmuskip\thickmuskip\thinmuskip\thickmuskip
   \relax#5#1\mkern-7mu%
   \cleaders\hbox{$#5\mkern-2mu#2\mkern-2mu$}\hfill
   \mathclap{#3}\mathclap{#2}%
   \cleaders\hbox{$#5\mkern-2mu#2\mkern-2mu$}\hfill
   \mkern-7mu#4$%
}
% Here's the idea: \<slashed>arrowfill@ should be a box containing
% some stretchable space that is the "middle of the arrow".  This
% space is created as a "leader" using \cleader<thing>\hfill, which
% fills an \hfill of space with copies of <thing>.  Here instead of
% just one \cleader, we use two, with the slash in between (and an
% extra copy of the filler, to avoid extra space around the slash).
\def\rightslashedarrowfill@{%
  \slashedarrowfill@\relbar\relbar\mapstochar\rightarrow}
\newcommand\xslashedrightarrow[2][]{%
  \ext@arrow 0055{\rightslashedarrowfill@}{#1}{#2}}
\mdef\hto{\xslashedrightarrow{}}
\mdef\htoo{\xslashedrightarrow{\quad}}
\let\xhto\xslashedrightarrow

%% To get a slashed arrow in XYmatrix, do
% \[\xymatrix{A \ar[r]|-@{|} & B}\]
%% To get it in diagxy, do
% \morphism/{@{>}|-*@{|}}/[A`B;p]

%% Here is an \hto for diagxy:
% \def\htopppp/#1/<#2>^#3_#4{\:%
% \ifnum#2=0%
%    \setwdth{#3}{#4}\deltax=\wdth \divide \deltax by \ul%
%    \advance \deltax by \defaultmargin  \ratchet{\deltax}{100}%
% \else \deltax #2%
% \fi%
% \xy\ar@{#1}|-@{|}^{#3}_{#4}(\deltax,0) \endxy%
% \:}%
% \def\htoppp/#1/<#2>^#3{\ifnextchar_{\htopppp/#1/<#2>^{#3}}{\htopppp/#1/<#2>^{#3}_{}}}%
% \def\htopp/#1/<#2>{\ifnextchar^{\htoppp/#1/<#2>}{\htoppp/#1/<#2>^{}}}%
% \def\htoop/#1/{\ifnextchar<{\htopp/#1/}{\htopp/#1/<0>}}%
% \def\hto{\ifnextchar/{\htoop}{\htoop/>/}}%

% LABELED ISOMORPHISMS
\def\xiso#1{\mathrel{\mathrlap{\smash{\xto[\smash{\raisebox{1.3mm}{$\scriptstyle\sim$}}]{#1}}}\hphantom{\xto{#1}}}}
\def\toiso{\xto{\smash{\raisebox{-.5mm}{$\scriptstyle\sim$}}}}

% SHADOWS
\def\shvar#1#2{{\ensuremath{%
  \hspace{1mm}\makebox[-1mm]{$#1\langle$}\makebox[0mm]{$#1\langle$}\hspace{1mm}%
  {#2}%
  \makebox[1mm]{$#1\rangle$}\makebox[0mm]{$#1\rangle$}%
}}}
\def\sh{\shvar{}}
\def\scriptsh{\shvar{\scriptstyle}}
\def\bigsh{\shvar{\big}}
\def\Bigsh{\shvar{\Big}}
\def\biggsh{\shvar{\bigg}}
\def\Biggsh{\shvar{\Bigg}}

% TYPING JUDGMENTS
% Call this macro as \jd{x:A, y:B |- c:C}.  It adds (what I think is)
% appropriate spacing, plus auto-sized parentheses around each typing judgment.
\def\jd#1{\@jd#1\ej}
\def\@jd#1|-#2\ej{\@@jd#1,,\;\vdash\;\left(#2\right)}
\def\@@jd#1,{\@ifmtarg{#1}{\let\next=\relax}{\left(#1\right)\let\next=\@@@jd}\next}
\def\@@@jd#1,{\@ifmtarg{#1}{\let\next=\relax}{,\,\left(#1\right)\let\next=\@@@jd}\next}
% Here's a version which puts a line break before the turnstyle.
\def\jdm#1{\@jdm#1\ej}
\def\@jdm#1|-#2\ej{\@@jd#1,,\\\vdash\;\left(#2\right)}
% Make an actual comma that doesn't separate typing judgments (e.g. A,B,C : Type).
\def\cm{,}

%% SKIPIT in TikZ
% See http://tex.stackexchange.com/questions/3513/draw-only-some-segments-of-a-path-in-tikz
\long\def\my@drawfill#1#2;{%
\@skipfalse
\fill[#1,draw=none] #2;
\@skiptrue
\draw[#1,fill=none] #2;
}
\newif\if@skip
\newcommand{\skipit}[1]{\if@skip\else#1\fi}
\newcommand{\drawfill}[1][]{\my@drawfill{#1}}

%% TODO: This \autoref in TAC doesn't work with figures (and anything
%% else other than theorems).

%%%% THEOREM-TYPE ENVIRONMENTS, hacked to
%%% (a) number all with the same numbers, and
%%% (b) have the right names.
%% The following code should work in TAC or out of it, and with
%% hyperref or without it.  In all cases, we use \label to label
%% things and \autoref to refer to them (ordinary \ref declines to
%% include names).  The non-hyperref label and reference hack is from
%% Mike Mandell, I believe.
\newif\ifhyperref
\@ifpackageloaded{hyperref}{\hyperreftrue}{\hyperreffalse}
\iftac
  %% In the TAC style, all theorems are actually subsections.  So
  %% the hyperref \autoref doesn't work and we have to use our own code
  %% in any case.  We also have to hook into the \state macros instead
  %% of \@thm since those are what know about the current theorem type.
  \let\your@state\state
  \def\state#1{\my@state#1}
  \def\my@state#1.{\gdef\currthmtype{#1}\your@state{#1.}}
  \let\your@staterm\staterm
  \def\staterm#1{\my@staterm#1}
  \def\my@staterm#1.{\gdef\currthmtype{#1}\your@staterm{#1.}}
  \let\@defthm\newtheorem
  \def\switchtotheoremrm{\let\@defthm\newtheoremrm}
  \def\defthm#1#2#3{\@defthm{#1}{#2}} % Ignore the third argument (for cleveref only)
  % The following allows us to use \cref for sections too, as if it
  % were cleveref.  (But not for subsections, and also not for
  % multiple references at once.)
  \let\your@section\section
  \def\section{\gdef\currthmtype{section}\your@section}
  % Start out \currthmtype as empty
  \def\currthmtype{}
  % In a bit, we're going to redefine \label so that \label{athm} will
  % also make a label "label@name@athm" which is the current value of
  % \currthmtype.  Now \autoref{athm} just adds a reference to this in
  % front of the reference.
  \ifhyperref
    \def\autoref#1{\ref*{label@name@#1}~\ref{#1}}
  \else
    \def\autoref#1{\ref{label@name@#1}~\ref{#1}}
  \fi
  % This has to go AFTER the \begin{document} because apparently
  % hyperref resets the definition of \label at that point.
  \AtBeginDocument{%
    % Save the old definition of \label
    \let\old@label\label%
    % Redefine \label so that \label{athm} will also make a label
    % "label@name@athm" which is the current value of \currthmtype.
    \def\label#1{%
      {\let\your@currentlabel\@currentlabel%
        \edef\@currentlabel{\currthmtype}%
        \old@label{label@name@#1}}%
      \old@label{#1}}
  }
  % TODO: This doesn't work for references to figures!
  \let\cref\autoref
\else\ifcref
  % Cleveref does most of it for us.
  \def\defthm#1#2#3{%
    %% Ensure all theorem types are numbered with the same counter
    \newaliascnt{#1}{thm}
    \newtheorem{#1}[#1]{#2}
    \aliascntresetthe{#1}
    %% This command tells cleveref's \cref what to call things
    \crefname{#1}{#2}{#3}% following brace must be on separate line to support poorman cleveref sed file
  }
  % \let\autoref\cref  % May want to use \autoref for xr-ed links
\else
  % In non-TAC styles without cleveref, theorems have their own counters and so the
  % hyperref \autoref works, if hyperref is loaded.
  \ifhyperref
    %% If we have hyperref, then we have to make sure all the theorem
    %% types appear to use different counters so that hyperref can tell
    %% them apart.  However, we want them actually to use the same
    %% counter, so we don't have both Theorem 9.1 and Definition 9.1.
    \def\defthm#1#2#3{% Ignore the third argument (for cleveref only)
      %% All types of theorems are number inside sections
      \newtheorem{#1}{#2}[section]%
      %% This command tells hyperref's \autoref what to call things
      \expandafter\def\csname #1autorefname\endcsname{#2}%
      %% This makes all the theorem counters actually the same counter
      \expandafter\let\csname c@#1\endcsname\c@thm}
  \else
    %% Without hyperref, we have to roll our own.  This code is due to
    %% Mike Mandell.  First, all theorems can now "officially" use the
    %% same counter.
    \def\defthm#1#2#3{\newtheorem{#1}[thm]{#2}} % Ignore the third argument (for cleveref only)
    %% Save the label- and theorem-making commands
    \ifx\SK@label\undefined\let\SK@label\label\fi
    \let\old@label\label
    \let\your@thm\@thm
    %% Save the current type of theorem whenever we start one
    \def\@thm#1#2#3{\gdef\currthmtype{#3}\your@thm{#1}{#2}{#3}}
    %% Start that out as empty
    \def\currthmtype{}
    %% Redefine \label so that \label{athm} defines, in addition to a
    %% label "athm" pointing to "9.1," a label "athm@" pointing to
    %% "Theorem 9.1."
    \def\label#1{{\let\your@currentlabel\@currentlabel\def\@currentlabel%
        {\currthmtype~\your@currentlabel}%
        \SK@label{#1@}}\old@label{#1}}
    %% Now \autoref just looks at "athm@" instead of "athm."
    \def\autoref#1{\ref{#1@}}
  \fi
  \let\cref\autoref
\fi\fi

%% Now the code that works in all cases.  Note that TAC allows the
%% optional arguments, but ignores them.  It also defines environments
%% called "theorem," etc.
\newtheorem{thm}{Theorem}[section]
\ifcref
  \crefname{thm}{Theorem}{Theorems}
\else
  \newcommand{\thmautorefname}{Theorem}
\fi
\defthm{cor}{Corollary}{Corollaries}
\defthm{prop}{Proposition}{Propositions}
\defthm{lem}{Lemma}{Lemmas}
\defthm{sch}{Scholium}{Scholia}
\defthm{assume}{Assumption}{Assumptions}
\defthm{claim}{Claim}{Claims}
\defthm{conj}{Conjecture}{Conjectures}
\defthm{hyp}{Hypothesis}{Hypotheses}
\iftac\switchtotheoremrm\else\theoremstyle{definition}\fi
\defthm{defn}{Definition}{Definitions}
\defthm{notn}{Notation}{Notations}
\iftac\switchtotheoremrm\else\theoremstyle{remark}\fi
\defthm{rmk}{Remark}{Remarks}
\defthm{warn}{Warning}{Warnings}
\defthm{eg}{Example}{Examples}
\defthm{egs}{Examples}{Examples}
\defthm{ex}{Exercise}{Exercises}
\defthm{ceg}{Counterexample}{Counterexamples}

\ifcref
  % Display format for sections
  \crefformat{section}{\S#2#1#3}
  \Crefformat{section}{Section~#2#1#3}
  \crefrangeformat{section}{\S\S#3#1#4--#5#2#6}
  \Crefrangeformat{section}{Sections~#3#1#4--#5#2#6}
  \crefmultiformat{section}{\S\S#2#1#3}{ and~#2#1#3}{, #2#1#3}{ and~#2#1#3}
  \Crefmultiformat{section}{Sections~#2#1#3}{ and~#2#1#3}{, #2#1#3}{ and~#2#1#3}
  \crefrangemultiformat{section}{\S\S#3#1#4--#5#2#6}{ and~#3#1#4--#5#2#6}{, #3#1#4--#5#2#6}{ and~#3#1#4--#5#2#6}
  \Crefrangemultiformat{section}{Sections~#3#1#4--#5#2#6}{ and~#3#1#4--#5#2#6}{, #3#1#4--#5#2#6}{ and~#3#1#4--#5#2#6}
  % Display format for appendices
  \crefformat{appendix}{Appendix~#2#1#3}
  \Crefformat{appendix}{Appendix~#2#1#3}
  \crefrangeformat{appendix}{Appendices~#3#1#4--#5#2#6}
  \Crefrangeformat{appendix}{Appendices~#3#1#4--#5#2#6}
  \crefmultiformat{appendix}{Appendices~#2#1#3}{ and~#2#1#3}{, #2#1#3}{ and~#2#1#3}
  \Crefmultiformat{appendix}{Appendices~#2#1#3}{ and~#2#1#3}{, #2#1#3}{ and~#2#1#3}
  \crefrangemultiformat{appendix}{Appendices~#3#1#4--#5#2#6}{ and~#3#1#4--#5#2#6}{, #3#1#4--#5#2#6}{ and~#3#1#4--#5#2#6}
  \Crefrangemultiformat{appendix}{Appendices~#3#1#4--#5#2#6}{ and~#3#1#4--#5#2#6}{, #3#1#4--#5#2#6}{ and~#3#1#4--#5#2#6}
  \crefformat{subappendix}{\S#2#1#3}
  \Crefformat{subappendix}{Section~#2#1#3}
  \crefrangeformat{subappendix}{\S\S#3#1#4--#5#2#6}
  \Crefrangeformat{subappendix}{Sections~#3#1#4--#5#2#6}
  \crefmultiformat{subappendix}{\S\S#2#1#3}{ and~#2#1#3}{, #2#1#3}{ and~#2#1#3}
  \Crefmultiformat{subappendix}{Sections~#2#1#3}{ and~#2#1#3}{, #2#1#3}{ and~#2#1#3}
  \crefrangemultiformat{subappendix}{\S\S#3#1#4--#5#2#6}{ and~#3#1#4--#5#2#6}{, #3#1#4--#5#2#6}{ and~#3#1#4--#5#2#6}
  \Crefrangemultiformat{subappendix}{Sections~#3#1#4--#5#2#6}{ and~#3#1#4--#5#2#6}{, #3#1#4--#5#2#6}{ and~#3#1#4--#5#2#6}
  % Display format for parts
  \crefname{part}{Part}{Parts}
  % Display format for figures
  \crefname{figure}{Figure}{Figures}
\fi


% \qedhere for TAC
\iftac
  \let\qed\endproof
  \let\your@endproof\endproof
  \def\my@endproof{\your@endproof}
  \def\endproof{\my@endproof\gdef\my@endproof{\your@endproof}}
  \def\qedhere{\tag*{\endproofbox}\gdef\my@endproof{\relax}}
\fi

% Make the optional arguments to TAC's \proof behave like everyone else's
\iftac
  \def\pr@@f[#1]{\subsubsection*{\sc #1.}}
\fi

% How to get QED symbols inside equations at the end of the statements
% of theorems.  AMS LaTeX knows how to do this inside equations at the
% end of *proofs* with \qedhere, and at the end of the statement of a
% theorem with \qed (meaning no proof will be given), but it can't
% seem to combine the two.  Use this instead.
\def\thmqedhere{\expandafter\csname\csname @currenvir\endcsname @qed\endcsname}

% Number numbered lists as (i), (ii), ...
\ifbeamer\else
  \renewcommand{\theenumi}{(\roman{enumi})}
  \renewcommand{\labelenumi}{\theenumi}
\fi

% Left margins for enumitem
\ifbeamer\else
  \setitemize[1]{leftmargin=2em}
  \setenumerate[1]{leftmargin=*}
\fi

% Also number formulas with the theorem counter
\iftac
  \let\c@equation\c@subsection
\else
  \let\c@equation\c@thm
\fi
\numberwithin{equation}{section}

% Only show numbers for equations that are actually referenced (or
% whose tags are specified manually) - uses mathtools.  All equations
% need to be referenced with \eqref, not \ref, for this to work!
\ifcref\else
  \@ifpackageloaded{mathtools}{\mathtoolsset{showonlyrefs,showmanualtags}}{}
\fi

% GREEK LETTERS, ETC.
\alwaysmath{alpha}
\alwaysmath{beta}
\alwaysmath{gamma}
\alwaysmath{Gamma}
\alwaysmath{delta}
\alwaysmath{Delta}
\alwaysmath{epsilon}
\mdef\ep{\varepsilon}
\alwaysmath{zeta}
\alwaysmath{eta}
\alwaysmath{theta}
\alwaysmath{Theta}
\alwaysmath{iota}
\alwaysmath{kappa}
\alwaysmath{lambda}
\alwaysmath{Lambda}
\alwaysmath{mu}
\alwaysmath{nu}
\alwaysmath{xi}
\alwaysmath{pi}
\alwaysmath{rho}
\alwaysmath{sigma}
\alwaysmath{Sigma}
\alwaysmath{tau}
\alwaysmath{upsilon}
\alwaysmath{Upsilon}
\alwaysmath{phi}
\alwaysmath{Pi}
\alwaysmath{Phi}
\mdef\ph{\varphi}
\alwaysmath{chi}
\alwaysmath{psi}
\alwaysmath{Psi}
\alwaysmath{omega}
\alwaysmath{Omega}
\let\al\alpha
\let\be\beta
\let\gm\gamma
\let\Gm\Gamma
\let\de\delta
\let\De\Delta
\let\si\sigma
\let\Si\Sigma
\let\om\omega
\let\ka\kappa
\let\la\lambda
\let\La\Lambda
\let\ze\zeta
\let\th\theta
\let\Th\Theta
\let\vth\vartheta
\let\Om\Omega

%% Include or exclude solutions
% This code is basically copied from version.sty, except that when the
% solutions are included, we put them in a `proof' environment as
% well.  To include solutions, say \includesolutions; to exclude them
% say \excludesolutions.
% \begingroup
% 
% \catcode`{=12\relax\catcode`}=12\relax%
% \catcode`(=1\relax \catcode`)=2\relax%
% \gdef\includesolutions(\newenvironment(soln)(\begin(proof)[Solution])(\end(proof)))%
% \gdef\excludesolutions(%
%   \gdef\soln(\@bsphack\catcode`{=12\relax\catcode`}=12\relax\soln@NOTE)%
%   \long\gdef\soln@NOTE##1\end{soln}(\solnEND@NOTE)%
%   \gdef\solnEND@NOTE(\@esphack\end(soln))%
% )%
% \endgroup

\makeatother

% Local Variables:
% mode: latex
% TeX-master: ""
% End:

\usepackage[status=draft]{fixme}
\title{Unifying constructive general topology}
\author{Toby Bartels and Michael Shulman}
\def\R{\mathbb{R}}
\def\Re{\overline{\mathbb{R}}}
\def\Rp{[0,\infty]}
%\def\Rpf{[0,\infty)}
\def\F{\mathcal{F}}
\def\apart{\mathrel{\#}}
\begin{document}
\maketitle

\section{Introduction}
\label{sec:intro}

\section{Metric, gauge, and prometric spaces}
\label{sec:metric}

Arguably, the goal of general topology is to describe ``spaces'' that are not metrizable but share many attributes of metric spaces.
Thus, it is worth recalling the basic definitions and properties of metric spaces in a constructive context.
We write $\Re$ for the set of extended Dedekind real numbers; these are pairs $(L,U)$ of sets of rational numbers such that:
\begin{enumerate}
\item if $x\in L$ and $y\in U$ then $x<y$,
\item $x\in L$ if and only if there is a $y\in L$ with $x<y$ (roundedness),
\item $y\in U$ if and only if there is an $x\in U$ with $x<y$ (roundedness), and
\item if $x<y$ then either $x\in L$ or $y\in U$ (locatedness).
\end{enumerate}
If we also required $L$ and $U$ to be inhabited, we would get the set $\R$ of ordinary Dedekind real numbers.
In classical mathematics, $\Re$ differs from $\R$ only by including $+\infty = (\mathbb{Q},\emptyset)$ and $-\infty = (\emptyset,\mathbb{Q})$.
Constructively, we can say that $\R = \setof{ x\in \Re | -\infty < x < \infty }$.
We also write
\[\Rp = \setof{x\in\Re | x\ge 0 }.\]

\begin{defn}
  An \textbf{extended quasi-pseudo-metric space} is a set $X$ together with a function $d:X\times X \to \Rp$ such that
  \begin{enumerate}
  \item $d(x,x)=0$ for all $x\in X$ (reflexivity), and
  \item $d(x,z)\le d(x,y)+d(y,z)$ for all $x,y,z\in X$ (transitivity, or the triangle inequality).
  \end{enumerate}
  A \textbf{quasi-pseudo-metric space} is an extended quasi-pseudo-metric space such that
  \begin{enumerate}[resume]
  \item $d(x,y)<\infty$ for all $x,y\in X$ (finite distances).
  \end{enumerate}
  A \textbf{pseudo-metric space} is a quasi-pseudo-metric space such that
  \begin{enumerate}[resume]
  \item $d(x,y)=d(y,x)$ for all $x,y\in X$ (symmetry).
  \end{enumerate}
  A \textbf{metric space} is a pseudo-metric space such that
  \begin{enumerate}[resume]
  \item if $d(x,y)=0$ then $x=y$ (separation).
  \end{enumerate}
\end{defn}

Most of the above terminology is so well-established that we have not attempted to depart from it, but despite their derogatory-sounding name it is really the extended quasi-pseudo-metric spaces that are the fundamental notion.
The additional requirements imposed on a pseudo-metric or metric space are really \emph{separation axioms}, and experience in general topology has shown that a better category is obtained by only imposing separation axioms when necessary rather than including them in definitions.
And allowing infinite distances is arguably of little import topologically.
Moreover, Lawvere~\cite{lawvere:metric-spaces} exhibited extended quasi-pseudo-metric spaces as categories enriched over $(\Rp,\ge,+)$, placing them in a wider context.

In fact, most topological constructions on metric spaces can easily be generalized to the following structures.

\begin{defn}
  An \textbf{(extended) (quasi-)gauge space} is a set $X$ equipped with a set (called a \textbf{gauge}) whose elements are (extended) (quasi-)pseudo-metrics on $X$ (called \textbf{gauging distances}) such that:
  \begin{enumerate}
  \item there exists some gauging distance (nullary filtration), and
  \item if $d_1$ and $d_2$ are gauging distances, there is a gauging distance $d_3$ such that $d_1(x,y)\le d_3(x,y)$ and $d_2(x,y)\le d_3(x,y)$ for all $x,y\in X$ (binary filtration).
  \end{enumerate}
  A gauge space is \textbf{separated} if
  \begin{enumerate}[resume]
  \item for any $x,y\in X$, if $d(x,y)=0$ for all gauging distances $d$, then $x=y$.
  \end{enumerate}
\end{defn}

In other words, a gauge is a filterbase in the poset of (extended quasi-)pseudo-metrics.
Note that the qualifiers ``extended'' and ``quasi-'' carry over from the metrics to the gauge, but all the metrics in a gauge space are automatically assumed to be pseudo; separatedness is named as an additional property.

Of course, any (extended quasi-)pseudo-metric space defines an (extended quasi-)gauge space with exactly one gauging distance.
The essential difference between gauge spaces and metric spaces is that gauge spaces need not be ``countable'' in any of the senses that metric spaces must.
For instance, any family of gauge spaces $\setof{X_k}_{k\in K}$ has a product $\prod_k X_k$ with gauging distances of the form $d(x,y) = \max(d_1(\pi_{k_1}(x),\pi_{k_1}(y)),\dots,d_n(\pi_{k_n}(x),\pi_{k_n}(y)))$ where each $d_i$ is a gauging distance on $X_{k_i}$.
An analogous construction for metric spaces only works if $K$ is finite (a slightly subtler one works if it is countable).

Yet more general than gauge spaces are prometric spaces~\cite{cht:one-setting}:

\begin{defn}
  A \textbf{prometric space} is a set $X$ equipped with a set of functions $d:X\times X\to\Rp$, called \textbf{distances}, such that:
  \begin{enumerate}
  \item there exists a distance (nullary filtration),
  \item if $d_1$ and $d_2$ are distances, there is a distance $d_3$ such that $d_1(x,y)\le d_3(x,y)$ and $d_2(x,y)\le d_3(x,y)$ for all $x,y\in X$ (binary filtration),
  \item $d(x,x)=0$ for every distance and all $x\in X$ (reflexivity),
  \item for any distance $d_1$ there is a distance $d_2$ such that $d_1(x,z)\le d_2(x,y)+d_2(y,z)$ for all $x,y,z\in X$ (transitivity, or the triangle inequality).
  \end{enumerate}
  A prometric space is \textbf{symmetric} if
  \begin{enumerate}[resume]
  \item for any distance $d_1$ there is a distance $d_2$ such that $d_1(x,y)\le d_2(y,x)$ for all $x,y\in X$ (symmetry).
  \end{enumerate}
  It is \textbf{separated} if
  \begin{enumerate}[resume]
  \item for any $x,y\in X$, if $d(x,y)=0$ for all distances $d$, then $x=y$.
  \end{enumerate}
\end{defn}

Since the terminology for prometric spaces is not well-established, we follow~\cite{cht:one-setting} in taking the ``extended quasi'' case to be the default, with symmetry stated explicitly whenever assumed.
Obviously any extended quasi-gauge is also a prometric.
We do not know any important examples of the extra generality afforded by prometric spaces, but they seem a natural level of generality; everything we have to say about metric spaces or gauge spaces in this paper applies just as well to prometric ones.

On the other hand, in constructive mathematics a very different (and much more general) notion is obtained by allowing distances to take values in the (nonnegative extended) \emph{upper} real numbers (sets $U$ of rational numbers satisfying the single axiom ``$y\in U$ if and only if there is an $x\in U$ with $x<y$'', plus $0\notin U$ for nonnegativity).
In this case we will speak of an \textbf{upper prometric space}.
If necessary, we distinguish ordinary real numbers from upper ones by calling them \textbf{located}, and ordinary prometric spaces from upper ones by calling them \textbf{decomposable}.

The main difference is that unlike located real numbers, upper real numbers constructively admit arbitrary infima.
Thus, for instance, any set $X$ admits a \emph{discrete} upper prometric with one distance defined by $d(x,y) = \inf \setof{ t | t=0 \land x=y }$, so that $d(x,x)=0$ but $d(x,y)=+\infty$ if $x\neq y$.
This distance takes real values if and only if $X$ has decidable equality, since if $d(x,y)$ were real then it would be either $<2$ (in which case $x=y$) or $>1$ (in which case $x\neq y$).
As we will see, locatedness of distances is crucial for the special behavior of (pro)metric spaces among arbitrary spaces in constructive mathematics.

We now review some of the fundamental concepts from metric spaces that one intends to capture for more general spaces, simultaneously generalizing them to prometric ones.
In the following, $X$ and $Y$ denote (perhaps upper) prometric spaces; $\ep$ and $\de$ are assumed to range over $\Rp$; and $d_X$ and $d_Y$ are assumed to range over distances on $X$ and $Y$ respectively.
We begin with continuity of functions.

\begin{itemize}
\item A function $f:X\to Y$ is \textbf{continuous} if for all $x\in X$, $d_Y$, and $\ep>0$ there are $d_X$ and $\de>0$ such that for all $y\in X$, if $d_X(x,y)<\de$ then $d_Y(f(x),f(y))<\ep$.
\item Similarly, $f:X\to Y$ is \textbf{uniformly continuous} if for all $d_Y$ and $\ep>0$ there are $d_X$ and $\de>0$ such that for all $x,y\in X$, if $d_X(x,y)<\de$ then $d_Y(f(x),f(y))<\ep$.
\item We will say $f:X\to Y$ is \textbf{cocontinuous}\fxwarning{Does this have a standard name?} if for all $x\in X$, $d_Y$, and $\ep>0$ there are $d_X$ and $\de>0$ such that for all $y\in X$, if $d_Y(f(x),f(y))>\ep$ then $d_X(x,y)>\de$.
\item Similarly, $f:X\to Y$ is \textbf{uniformly cocontinuous} if for all $d_Y$ and $\ep>0$ there are $d_X$ and $\de>0$ such that for all $x,y\in X$, if $d_Y(f(x),f(y))>\ep$ then $d_X(x,y)>\de$.
\end{itemize}

If $X$ is decomposable and $f$ is (uniformly) continuous, then it is also (uniformly) cocontinuous.
For given $d_Y$ and $\ep$, let $d_X$ and $\de$ be as in (uniform) continuity; then if $d_Y(f(x),f(y)) > \ep$, we cannot have $d_X(x,y) < \de$, so we must have $d_X(x,y) > \frac{\de}{2}$.

On the other hand, if $Y$ is decomposable and $f$ is (uniformly) cocontinuous, then it is also (uniformly) continuous.
For given $d_Y$ and $\ep$, let $d_X$ and $\de$ be such that $d_Y(f(x),f(y))>\frac{\ep}{2} \Rightarrow d_X(x,y)>\de$; then if $d_X(x,y)<\de$, we cannot have $d_Y(f(x),f(y))>\frac{\ep}{2}$, so we must have $d_Y(f(x),f(y))<\ep$.

Thus, for prometric spaces continuity and cocontinuity coincide; but this need not be the case for upper prometric spaces.

We continue with notions of convergence and completeness.

\begin{itemize}
\item $X$ is \textbf{totally bounded} if for all $d_X$ and $\ep>0$ there is a finite set $F\subseteq X$ such that for all $y\in X$ there is an $x\in F$ such that $d_X(x,y)<\ep$.
\item A sequence $\setof{x_n}$ in $X$ \textbf{converges} to $x\in X$ if for every $d_X$ and $\ep>0$ there is an $N$ such that for any $n$, if $n>N$ then $d_X(x,x_n)<\ep$.
\item More generally, a filter $\F$ in $X$ \textbf{converges} to $x\in X$ if for every $d_X$ and $\ep>0$ there is an $A\in\F$ such that for all $y\in A$ we have $d_X(x,y)<\ep$.
\item A sequence $\setof{x_n}$ in $X$ is \textbf{Cauchy} if for every $d_X$ and $\ep>0$ there is an $N$ such that for any $n,m$, if $n,m>N$ then $d_X(x_n,x_m)<\ep$.
\item A filter $\F$ in $X$ is \textbf{Cauchy} if for every $d_X$ and $\ep>0$ there is an $A\in\F$ such that for all $x,y\in A$ we have $d_X(x,y)<\ep$.
\item $X$ is \textbf{complete} if every Cauchy filter converges to some point.
\end{itemize}

For metric spaces, it suffices to consider convergent and Cauchy \emph{sequences}, but in the gauge and prometric cases we must deal with filters (or nets).
Note that the above definitions of Cauchy filter and complete space are only really sensible in the symmetric case; we will return to this later.\fxnote{Do that}

In classical mathematics, a metric space is \emph{compact} if and only if it is complete and totally bounded.
Constructively, this is not so; thus one generally either uses ``complete and totally bounded'' as a replacement for ``compact'', or passes to locales instead of spaces (see \cref{sec:locales}).

% Classically, one can \emph{complete} a metric space by equipping a set of equivalence classes of Cauchy sequences with a metric.

We now embark on a process of generalization whose goal is to remove the ``non-topological'' information contained in a prometric, and express the result in an intrinsic way.
In particular, prometric spaces that are topologically isomorphic (by continuous functions), or in some cases uniformly isomorphic (by uniformly continuous functions), ought to give rise to ``topological'' structures that are actually indistinguishable.
In general the way we do this is by considering \emph{relations} between points or subsets rather than distances between them.


\section{Point-point relations: order and apartness}
\label{sec:point-point}
\label{sec:order}

There are two important relations between points in a prometric space.

\begin{defn}
  Let $x,y\in X$ be points of an upper prometric space.
  \begin{itemize}
  \item We say $x\le y$ if $d(y,x)=0$ for all distances $d$.\fxwarning{ordering convention?}
  \item We say $x<y$ if $d(x,y)>0$ for some distance $d$.
  \end{itemize}
\end{defn}

Reflexivity and transitivity (i.e.\ the triangle inequality) of an upper prometric imply that $\le$ is a \textbf{preorder}, i.e.\ a reflexive and transitive relation, and that $<$ is irreflexive ($x\not<x$).
As we will see later, the preorder $\le$ is an instance of the \emph{specialization order} underlying a topology.
(The notation for $<$ is somewhat misleading; in particular it is rarely transitive.)

If $X$ is decomposable (that is, it is a prometric space rather than just an upper prometric space), then $<$ is also a \emph{comparison}, i.e.\ if $x<z$ then for any $y$ we have $x<y$ or $y<z$.
For if $d(x,z)>0$, we have another distance $d'$ with $d'(x,y)+d'(y,z)\ge d(x,z)$.
By locatedness, $d'(x,y)$ and $d'(y,z)$ are both either $>0$ or $<\frac12 d(x,z)$; thus either $x<y$ or $y<z$ or $d'(x,y)+d'(y,z) < d(x,z)$, but the latter case is a contradiction.

Finally, if $X$ is symmetric, then $\le$ and $<$ are both symmetric relations.
A reflexive, transitive, and symmetric relation is of course called an \textbf{equivalence relation}; we will write it as $x\approx y$.
An irreflexive, symmetric, comparison relation is called an \textbf{apartness relation}, usually written $x\apart y$.

For \emph{located} real numbers, $\neg(a<b)$ is equivalent to $b\le a$.
(For upper real numbers, we define $U_1 < U_2$ if $U_1\setminus U_2$ is inhabited, and $U_2\le U_1$ if $U_1 \subseteq U_2$; neither of these is the negation of the other.)
Thus, if $X$ is decomposable, $x\approx y$ is equivalent to $\neg(x\apart y)$.
However, even in this case, $\apart$ cannot constructively be expressed in terms of $\approx$; to say $x\apart y$ is stronger than $x\neq y$.

If $X$ is also separated, then $x\approx y$ (and even $x\le y$, in the non-symmetric case) implies $x=y$.
An apartness relation such that $\neg(x\apart y)$ implies $x=y$ is called \textbf{tight}.
In particular, the equality relation of a separated prometric space is $\neg\neg$-stable, for $\neg\neg(x=y)$ implies $\neg(x\apart y)$, hence $x\approx y$ and thus $x=y$.

Finally, we observe that the point-point relations underlying a prometric space vary appropriately with continuous functions, so that they really are a topological invariant.
On one hand, if $f:X\to Y$ is a continuous map of upper prometric spaces, then $x\le y$ in $X$ implies $f(x)\le f(y)$ in $Y$, i.e.\ $f$ induces a map of posets (and hence a map of equivalence relations, if $X$ and $Y$ are symmetric).
Of course, in the separated case this is trivial.

On the other hand, if $f:X\to Y$ is \emph{cocontinuous}, then $f(x)<f(y)$ in $Y$ implies $x<y$ in $X$.
In general, a function between sets with apartness relations such that $f(x)\apart f(y)$ implies $x\apart y$ is called \textbf{strongly extensional}.
Thus, any cocontinuous function between symmetric upper prometric spaces (hence any continuous function between symmetric prometric spaces) is strongly extensional.


\section{Point-set relations: topology}
\label{sec:point-set}
\label{sec:topology}

  % Define these when we introduce their abstract versions.
% \item $U\subseteq X$ is a \textbf{neighborhood} of $x\in U$ if there is an $\ep>0$ such that for all $y$, if $d(x,y)<\ep$ then $y\in U$.
% \item $U\subseteq X$ is \textbf{open} if it is a neighborhood of each of its points, i.e.\ for all $x\in U$ there is an $\ep>0$ such that for all $y$, if $d(x,y)<\ep$ then $y\in U$.
% \item $C\subseteq X$ is \textbf{weakly closed} if whenever for all $\ep>0$ there is a $y\in C$ with $d(x,y)<\ep$, then $x\in C$.
%   It is \textbf{strongly closed} if $C=X\setminus U$ for some open set $U$.
% \item The \textbf{weak closure} of $C\subseteq X$ is the set of all $x$ such that for all $\ep>0$ there is a $y\in C$ with $d(x,y)<\ep$.
%   Thus $C$ is weakly closed iff it is equal to its weak closure.
% \item $A\subseteq X$ is \textbf{strongly dense} if for any $x\in X$ and $\ep>0$, there is an $a\in A$ with $d(x,a)<\ep$.
%   It is \textbf{weakly dense} if there does not exist an $x$ and an $\ep>0$ such that $d(x,y)<\ep$ implies $y\notin A$.


\section{Filters of point-point relations: uniformity}
\label{sec:uniformity}


\section{Set-set relations: proximity}
\label{sec:set-set}
\label{sec:proximity}




\section{Located subspaces}
\label{sec:located}

\begin{defn}
  A subset $A\subseteq X$ is \textbf{(metrically) located} if for all $x\in X$, the set $\setof{ d(x,y) | y\in A }$ has an greatest lower bound, written $d(x,A)$.
  \fxwarning{Which direction does this go in the quasi case?}
\end{defn}

Classically, the poset $\Rp$ is a complete lattice, and thus every subset is located.
(Even the empty set: $d(x,\emptyset)=\infty$.
This is one good reason for allowing infinite distances.)
However, constructively there are ``wild'' subsets such as $\setof{x\in X | P}$, for some undecidable proposition $P$, that cannot be shown to be located in general.
\fxnote{Mention some theorems about located subsets.}

Note: approach spaces have a distance between points and sets, definable from a metric space by infima.
Constructively this produces only an upper-real-valued distance, unless all subsets are located.
Presumably there are ``pro-approach spaces'' that arise from prometric spaces, or at least from gauge spaces.
Upper pro-approach spaces also give rise to syntopogenous spaces, but not decomposable ones.


\bibliographystyle{alpha}
\bibliography{syntop}

\end{document}
